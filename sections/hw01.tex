\documentclass[../hw_sols.tex]{subfiles}
\setlist[description]{style = unboxed, leftmargin = 0.55cm}


\begin{document}


%%=============================================================================
%%============================== PROBLEM 3.1 ==================================
%%=============================================================================

\subsection*{Problem 3.1}

A generalization of the Caesar cipher, known as the affine Caesar cipher, has 
the following form: For each plaintext letter $p$, substitute the ciphertext 
letter $C$:
	$$C = E([a,b], p) = (ap + b) \tmod{26}$$
A basic requirement of any encryption algorithm is that it be one-to-one. 
That is, if $p \neq q$, then $E(k,p) \neq E(k,q)$. Otherwise, decryption is 
impossible, because more than one plaintext character maps into the same 
ciphertext character. The affine Caesar cipher is not one-to-one for all 
values of $a$. For example, for $a = 2$ and $b = 3$, then \newline
$E([a,b], 0) = E([a,b], 13) = 3$.

\begin{description}

%=== PART A ====
\item[a.] Are there any limitations on the value of $b$? Explain why or why 
not. [3 points]

%=== SOLUTION ====
\begin{solution}
Addition by $b$ represents a shift substitution. Therefore, impose 
$b \in \mathbb{Z}_{26}$ so each shift is unique.
\end{solution}

%=== PART B ====
\item[b.] Determine which values of $a$ are not allowed. [3 points]

%=== SOLUTION ===
\begin{solution}
The decryption algorithm is easy to construct.
	$$D([a,b],C) := a^{-1}(C-b) \tmod{26} = p$$
For $a^{-1} \tmod{26}$ to exists, we require $\gcd(a,26) = 1$. Hence, $a$ 
cannot be even or 13.
\end{solution}

\end{description}


\newpage


%%=============================================================================
%%============================== PROBLEM 3.8 ==================================
%%=============================================================================

\subsection*{Problem 3.8}

A disadvantage of the general monoalphabetic cipher is that both sender and 
receiver must commit the permuted cipher sequence to memory. A common 
technnique for avoiding this is to use a keyword from which the cipher 
sequence can be generated. For example, using the keyword CRYPTO, write out 
the keyword followed by unused letters in normal order and match this against 
the plaintext letters:
\begin{verbatim}
    plain:  a b c d e f g h i j k l m n o p q r s t u v w x y z
    CIPHER: C R Y P T O A B D E F G H I J K L M N Q S U V W X Z
\end{verbatim}

\noindent If it is felt that this process does not produce sufficient mixing, 
write the remaining letters on successive lines and then generate the sequence 
by reading down the columns:
\begin{verbatim}
            C R Y P T O
            A B D E F G
            H I J K L M
            N Q S U V W
            X Z
\end{verbatim}

\noindent This yields the sequence:
\begin{verbatim}
    C A H N X R B I Q Z Y D J S P E K U T F L V O G M W
\end{verbatim}

% TODO: Option [page=#] doesn't work.
\noindent Such a system is used in the example in 
\href{run:../Cryptography and Network Security.pdf}{Section 3.2} (the one that 
begins "it was disclosed yesterday"). Determine the keyword. [3 points]

%=== SOLUTION ====
\begin{solution}

From the decrypt, we know the correspondence below:
\begin{verbatim}
    plain:  a b c d e f g h i j k l m n o p q r s t u v w x y z
    CIPHER: S A H V P B J W U ? ? X T D M Y ? E O Z I F Q ? G ?	
\end{verbatim}

\noindent The keyword must have a length of 6 so that \verb|A| and \verb|B| 
line up horizontally.

\begin{minipage}{0.3\linewidth}
	\begin{verbatim}
            S P U T ? I ?
            A B ? D E F G
            H J ? M O Q ?
            V W X Y Z    
	\end{verbatim}
\end{minipage}
\qquad {\Huge $\Rightarrow$} \qquad
\begin{minipage}{0.3\linewidth}
	\begin{verbatim}
    S P U T N I K
    A B C D E F G
    H J L M O Q R
    V W X Y Z
	\end{verbatim}
\end{minipage}

\noindent With little effort, we see the keyword is \verb|SPUTNIK|.

\end{solution}


\newpage


%%=============================================================================
%%============================== PROBLEM 3.9 ==================================
%%=============================================================================

\subsection*{Problem 3.9}

When the PT-109 American patrol boat, under the command of Lieutenant John F. 
Kennedy, was sunk by a Japanese destroyer, a message was received at an 
Australian wireless station in Playfair code:
\begin{verbatim}
            KXJEY UREBE ZWEHE WRYTU HEYFS
            KREHE GOYFI WTTTU OLKSY CAJPO
            BOTEI ZONTX BYBNT GONEY CUZWR
            GDSON SXBOU YWRHE BAAHY USEDQ
\end{verbatim}

\noindent The key used was \textit{royal new zealand navy}. Decrypt the 
message. Translate \verb|TT| into tt. [3points]

%=== SOLUTION ====
\begin{solution}

Begin by creating the $5 \times 5$ Playfair matrix from the known keyword.
	\begin{center}
	\begin{tabular}{ | c | c | c | c | c | }
		\hline \rowcolor{cyan!40}
		R & O & Y & A & L \\
		\hline \rowcolor{cyan!40}
		N & E & W & Z & D \\
		\hline
		\cellcolor{cyan!40} V & B & C & F & G \\
		\hline
		H & I/J & K & M & P \\
		\hline
		Q &   S & T & U & X \\
		\hline
	\end{tabular}
	\end{center}

\noindent Then, we can break up the ciphertext into digrams and decrypt each 
by applying the algorithm in reverse.
\begin{verbatim}
	CIPHER: KX JE YU RE BE ZW E H  EW RY TU  H E YF S K  RE  H E GO YF IW TT TU OL  K S YC AJ
	plain:  pt bo at on eo we ni/j ne lo st i/jn ac ti/j on i/jn bl ac ke tt st ra i/jt tw om

	CIPHER: P O  BO TE IZ ON TX BY BN TG ON EY CU ZW RG DS ON SX BO UY WR  H E BA AH YU  S E DQ
	plain:  i/jl es sw me re su co ve xc re wo ft we lv ex re qu es ta ny i/jn fo rm at i/jo nx
\end{verbatim}

\noindent When rearranging the cipher, we can ignore extra x's to get the 
plaintext:
\begin{verbatim}
	    pt boat one owe nine lost in action in blackett strait two miles
	    sw meresu cove crew of twelve request any information
\end{verbatim}

\end{solution}


\newpage


%%=============================================================================
%%============================== PROBLEM 3.11 =================================
%%=============================================================================

\subsection*{Problem 3.11}

\begin{description}

%=== PART A ====
\item[a.] Using this Playfair matrix:
	\begin{center}
	\begin{tabular}{ | c | c | c | c | c | }
		\hline
		M & F & H & I/J & K \\
		\hline
		U & N & O &   P & Q \\
		\hline
		Z & V & W &   X & Y \\
		\hline
		E & L & A &   R & G \\
		\hline
		D & S & T &   B & C \\
		\hline
	\end{tabular}
	\end{center}

\noindent Encrypt this message:
	$$\text{Must see you over Cadogan West. Coming at once.}$$
\textit{Note:} This message is from the Sherlock Holmes story, The 
Adventure of the Bruce-Partingon plans. [3 points]

%=== SOLUTION ====
\begin{solution}
Split the plaintext into digrams and add pad with an 'x' to encrypt as shown 
below.
\begin{verbatim}
    plain:  mu st se ey ou ov er ca do ga nw es tc om in ga to nc ex
    CIPHER: UZ TB DL GZ PN NW LG TG TU ER VO LD BD UH FP ER HW QS RZ
\end{verbatim}
\end{solution}


%=== PART B ====
\item[b.] Repeat part (a) using the keyword \textit{largest}. [3 points]

%=== SOLUTION ===
\begin{solution}

Using the keyword \textit{largest}, we construct the Playfair matrix below 
and encrypt.
	\begin{center}
	\begin{tabular}{ | c | c | c | c | c | }
		\hline \rowcolor{cyan!40}
		L & A & R & G & E \\
		\hline
		S \cellcolor{cyan!40} & T \cellcolor{cyan!40} & B & C & D\\
		\hline
		F & H & I/J & K & M\\
		\hline
		N & O &   P & Q & U\\
		\hline
		V & W &   X & Y & Z\\
		\hline
	\end{tabular}
	\end{center}

\begin{verbatim}
    CIPHER: UZ TB DL GZ PN NW LG TG TU ER OV DL BD UH PF ER HW QS RZ
\end{verbatim}

\end{solution}


%=== PART C ====
\item[c.] How do you account for the results of this problem? Can you 
generalize your conclusion? [3 points]

%=== SOLUTION ===
\begin{solution}
\newline  % Adding empty causes error.
\begin{minipage}{0.55\linewidth}
The Playfair matrix can be thought of as a torus $\mathcal{T}$ by folding and 
gluing the vertical edges to each other and the same with the horizontal edges. 
The figure to the right shows the two Playfair matrices from parts \textbf{a} 
and \textbf{b}, outlined in red and blue. Notice that they both generate 
$\mathcal{T}$. It makes sense that our ciphertexts for parts \textbf{a} and 
\textbf{b} were the same. If we wanted different ciphertexts, we'd need to 
swap rows or columns of the Playfair matrix. This wouldn't be equivalent to any 
shift and thus providing a new ciphertext.
\end{minipage}
\quad
\begin{minipage}{0.4\linewidth}
\begin{tikzpicture}
	\matrix [matrix of math nodes] (m)
	{
		Y & Z & V & W & X & Y & Z & V & W \\
		G & E & L & A & R & G & E & L & A \\
		C & D & S & T & B & C & D & S & T \\
		K & M & F & H & I/J & K & M & F & H \\
		Q & U & N & O & P & Q & U & N & O \\
		Y & Z & V & W & X & Y & Z & V & W \\
		G & E & L & A & R & G & E & L & A \\
		C & D & S & T & B & C & D & S & T \\
		K & M & F & H & I/J & K & M & F & H \\
	};

	\draw[color=red, thick] 
		(m-2-3.north west) -- (m-2-7.north east) -- (m-6-7.south east) 
		-- (m-6-3.south west) -- (m-2-3.north west);
	\draw[color=cyan, thick] 
		(m-4-2.north west) -- (m-4-6.north east) -- (m-8-6.south east) 
		-- (m-8-2.south west) -- (m-4-2.north west);
\end{tikzpicture}
\end{minipage}

\end{solution}

\end{description}


\newpage


%%=============================================================================
%%============================== PROBLEM 3.14 =================================
%%=============================================================================

\subsection*{Problem 3.14}

\begin{description}

%=== PART A ====
\item[a.] Encrypt the message "meet me at the usual place at ten rather than 
eight o clock" using the Hill cipher with the key 
$\begin{bmatrix} 7 & 3 \\ 2 & 5 \end{bmatrix}$
. Show your calculations and the result. [3 points]

%=== SOLUTION ===
\begin{solution}

Since the Hill cipher matrix $K \in \mathbb{Z}^{2 \times 2}$, we'll rewrite the 
plaintext into digrams and convert to numeric. We'll also pad the string with 
'x' at the end to achieve an even length.
\begin{verbatim}
    plain: me et me at th eu su al pl ac ea tt en ra th er th an ei gh to cl oc kx
\end{verbatim}

This gives us the following list of vectors $p_i$ for $1 \leq i \leq 24$:
\begin{align*}
	(12,4), (4,19), (12,4), (0,19), (19,7), (4,20),& 
		(18,20), (0,11), (15,11), (0,2), (4,0), (19,19), \\
	(4,13), (17,0), (19,7), (4,17), (19,7), (0,13),& 
		(4,8), (6,7), (19,14), (2,11), (14,2), (10,23).
\end{align*}

To encrypt, let $p_i$ be the $i^{\text{th}}$ row of matrix $P$ so that
\begin{align*}
	PK =& 
	\begin{bmatrix}
		12 & 4 & 12 & 0 & 19 & \dots & 14 & 10 \\
		4 & 19 & 4 & 19 & 7 & \dots & 2 & 23
	\end{bmatrix}^T
	\begin{bmatrix} 7 & 3 \\ 2 & 5 \end{bmatrix} 
	\tmod{26} \\
	\equiv& 
	\begin{bmatrix}
		14 & 14 & 14 & 12 & 17 & \dots & 24 & 12 \\
		4 & 3 & 4 & 17 & 14 & \dots & 0 & 15
	\end{bmatrix}^T 
	= \; C.
\end{align*}

We now convert each row $c_i$ of $C$ to alphabet characters to get the 
ciphertext below.
\begin{verbatim}
    CIPHER: OE OD OE MR QI KY WD XW EK CM PW CZ PZ RO AN SA EB FX KJ YA MP
\end{verbatim}

\end{solution}


%=== PART B ====
\item[b.] Show the calculations for the corresponding decryption of the 
ciphertext to recover the original plaintext. [3 points]

%=== SOLUTION ===
\begin{solution}

For decryption, compute $\det(K) = 29 \equiv 3 \tmod{26}$. Since 
$9 = 3^{-1} \tmod{26}$, we have
	$$K^{-1} = 9
	\begin{bmatrix} 5 & -3 \\ -2 & 7 \end{bmatrix} 
	\equiv
	\begin{bmatrix} 19 & 25 \\ 8 & 11 \end{bmatrix} 
	\tmod{26}.$$

To obtain the plaintext, we simply take the product	
\begin{align*}
	CK^{-1} =& 
	\begin{bmatrix}
		14 & 14 & 14 & 12 & 17 & \dots & 24 & 12 \\
		4 & 3 & 4 & 17 & 14 & \dots & 0 & 15
	\end{bmatrix}^T
	\begin{bmatrix} 19 & 25 \\ 8 & 11 \end{bmatrix}
	\tmod{26} \\
	\equiv& 
	\begin{bmatrix}
		12 & 4 & 12 & 0 & 19 & \dots & 14 & 10 \\
		4 & 19 & 4 & 19 & 7 & \dots & 2 & 23
	\end{bmatrix}
	= P.
\end{align*}

This recovers plaintext matrix $P$ from part \textbf{a}.

\end{solution}

\end{description}


\newpage


%%=============================================================================
%%============================== PROBLEM 3.18 =================================
%%=============================================================================

\subsection*{Problem 3.18}

\begin{description}

%=== PART B ====
\item[b.] Determine the inverse mod 26 of
	$$\begin{bmatrix}
		6  & 24 &  1 \\ 
		13 & 16 & 10 \\ 
		20 & 17 & 15
	\end{bmatrix}
	. \qquad\qquad \text{[3 points]}$$

%=== SOLUTION ===
\begin{solution}

Denote matrix $A$. To compute $A^{-1} \tmod{26}$, we first need the 
determinant.
	$$\det(A) = 441 \equiv 25 \tmod{26}$$

We can then find the cofactors of 
	$A^T = 
	\begin{bmatrix}
		 6 & 13 & 20 \\ 
		24 & 16 & 17 \\ 
		 1 & 10 & 15
	\end{bmatrix}$
as shown below.
	
\begin{multicols}{3}
	$C_{1,1} = (-1)^2(16 \cdot 15 - 10 \cdot 17) \\
	C_{2,1} = (-1)^3(13 \cdot 15 - 10 \cdot 20)  \\
	C_{3,1} = (-1)^4(13 \cdot 17 - \cdot 16 \cdot 20)$
	
	$C_{1,2} = (-1)^3(24 \cdot 15 - 1 \cdot 17) \\
	C_{2,2} = (-1)^4(6 \cdot 15 - 1 \cdot 20)   \\
	C_{3,2} = (-1)^5(6 \cdot 17 - 24 \cdot 20)$
	
	$C_{1,3} = (-1)^4(24 \cdot 10 - 1 \cdot 16) \\
	C_{2,3} = (-1)^5(6 \cdot 10 - 1 \cdot 13)   \\
	C_{3,3} = (-1)^6(6 \cdot 16 - 24 \cdot 13)$
\end{multicols}

This allows us to create the adjoint matrix
	$$\text{adj}(A) =
	\begin{bmatrix}
		 70 & -343 &  224 \\ 
		  5 &   70 &  -47 \\ 
		-99 &  378 & -216
	\end{bmatrix}.$$

Lastly, notice that $25 \cdot 25 \equiv 1 \tmod{26}$ so we take the product
	$$(\det(A))^{-1} \cdot \text{adj}(A) 
	= 25 \cdot \text{adj}(A) 
	\equiv 
	\begin{bmatrix}
		 8 &  5 & 10 \\ 
		21 &  8 & 21 \\ 
		21 & 12 & 8
	\end{bmatrix} 
	\tmod{26} = A^{-1}.$$

\end{solution}

\end{description}


\newpage


%%=============================================================================
%%============================== PROBLEM 3.19 =================================
%%=============================================================================

\subsection*{Problem 3.19}

Using the Vigenère cipher, encrypt the word "explanation" using the word 
"leg". [3 points]

%=== SOLUTION ===
\begin{solution}

Begin by repeating the keyword over the plaintext as shown below.
\begin{verbatim}
    key:    legleglegle
    plain:  explanation
\end{verbatim}

We can then perform the required shifts by taking the sum mod 26.
	\begin{center}
	\begin{tabular}{ *{12}{| c } | }
		\hline
		   key & 11 &  4 &  6 & 11 & 4 &  6 & 11 &  4 &  6 & 11 &  4 \\
		\hline
		 plain &  4 & 23 & 15 & 11 & 0 & 13 &  0 & 19 &  8 & 14 & 13 \\
		\hline
		CIPHER & 15 &  1 & 21 & 22 & 4 & 19 & 11 & 23 & 14 & 25 & 17 \\
		\hline
	\end{tabular}
	\end{center}

Encryption is completed by writing the numerical values into an alphabetic 
ciphertext.
\begin{verbatim}
    CIPHER: PBVWETLXOZR
\end{verbatim}

\end{solution}


\newpage


%%=============================================================================
%%============================== PROBLEM A ====================================
%%=============================================================================

\subsection*{Problem A}

A precursor to the ADFGVX cipher was the ADFGX cipher which used a table such 
as this:
	\begin{center}
	\begin{tabular}{ c | c c c c c }
		  & A &  D  & F & G & X \\
		\hline
		A & b &  t  & a & l & p \\
		D & d &  h  & o & z & k \\
		F & q &  f  & v & s & n \\
		G & g & i/j & c & u & x \\
		X & m &  r  & e & w & y 
	\end{tabular}
	\end{center}

Encrypt the phrase "neither do they spin" below. Use the grid on the left 
below to write the two-letter substitutions row-wise. Then rearrange the 
columns so that the column headers are in alphabetical order in the grid on 
the right.
	\begin{center}
	\begin{tabular}{ || c | c | c | c | c || }
		\hline
		\textbf{M} & \textbf{E} & \textbf{R} & \textbf{I} & \textbf{T} \\
		\hline & & & & \\ \hline & & & & \\	
		\hline & & & & \\ \hline & & & & \\
		\hline & & & & \\ \hline & & & & \\
		\hline & & & & \\ \hline
	\end{tabular}
	\hspace{2cm}
	\begin{tabular}{ || *{5}{m{0.3cm} | } | }
		\hline & & & & \\
		\hline & & & & \\ \hline & & & & \\	
		\hline & & & & \\ \hline & & & & \\
		\hline & & & & \\ \hline & & & & \\
		\hline & & & & \\ \hline
	\end{tabular}
	\end{center}

Write the ciphertext by reading out the grid on the right 
column-wise. [3 points]

%=== SOLUTION ===
\begin{solution}

Using the provided ADFGX table, fill out the two tables below.
	\begin{center}
	\begin{tabular}{ || c | c | c | c | c || }
		\hline
		\textbf{M} & \textbf{E} & \textbf{R} & \textbf{I} & \textbf{T} \\
		\hline
		F & X & X & F & G \\
		\hline
		D & A & D & D & D \\
		\hline
	    X & F & X & D & D \\
	    \hline
		A & D & F & A & D \\
		\hline
		D & D & X & F & X \\
		\hline
		X & F & G & A & X \\
		\hline
		G & D & F & X & \\
		\hline
	\end{tabular}
	\qquad {\Huge $\Rightarrow$} \qquad
	\begin{tabular}{ || c | c | c | c | c || }
		\hline
		\textbf{E} & \textbf{I} & \textbf{M} & \textbf{R} & \textbf{T} \\
		\hline
		X & F & F & X & G \\
		\hline
		A & D & D & D & D \\
		\hline
		F & D & X & X & D \\
		\hline
		D & A & A & F & D \\
		\hline
		D & F & D & X & X \\
		\hline
		F & A & X & G & X \\
		\hline
		D & X & G & F & \\
		\hline
	\end{tabular}
	\end{center}

Thus, we have the encryption of the phrase "neither do they spin" below.
\begin{verbatim}
    CIPHER: XAFDDFDFDDAFAXFDXADXGXDXFXGFGDDDXX
\end{verbatim}

\end{solution}


\newpage


%%=============================================================================
%%============================== PROBLEM B ====================================
%%=============================================================================

\subsection*{Problem B}

Decrypt the following permutation substitution cipher. [3 points]

% ciphertext from file
\VerbatimInput[xleftmargin=15pt]{scripts/hw01/b_cipher}

%=== SOLUTION ===
\begin{solution}

Begin by analyzing the frequency distribution using C.

% vscode snippet
\lstinputlisting[
	language = C, style = myCstyle,
	linerange = {31-38},
	numbersep = -10pt,     % workaround for removing 
	xleftmargin = -5pt     % tab from beginning
	]{hw01/show_outputs.c}

\begin{tikzpicture}
\begin{axis}[
	title = {Frequency Distribution},
	xmin = 0, xmax = 1,                           % extra space at ends
	xtick = data, tickwidth = 0pt,
	xticklabel style = {text height = 1.5ex},	  % extra space between bars
	ymin = 0, ymax = 40,
	ymajorgrids = true, grid style = dashed,
	ybar, bar width = 0.5cm,
	width = \textwidth, height = 0.3\textwidth,
	symbolic x coords = {0,a,b,c,d,e,f,g,h,i,j,k,l,m,n,o,p,q,r,s,t,u,v,w,x,y,z,1}
	]
	\addplot table [header=false] {hw01/b_frequency};  % values from file
\end{axis}
\end{tikzpicture}

The frequency distribution is very close to that of a monoalphabetic cipher. 
We can begin to take guesses for certain letters based on the distribution. 
Notice, there is an 11-letter word repeated twice with a 5-letter prefix.

\begin{BVerbatim}[commandchars=\\\{\}]

    CIPHER: EMGLOSUDCGDNCUSWYSFHNSFCYKDPUMLWGYICOXYSIPJCKQPKUGKMGOLICGINCGACKSNI
    plain:          e   e     t   te           e       e            e   e  e

    CIPHER: SACYKZSCKXECJCKSHYSXCGOIDPKZCNKSHICGIWYGKKGKGOLDSILKGOIUSIGLEDSPWZUG
    plain:    e  h e   e e      e      he     e                              h

    CIPHER: \textcolor{red}{FZCCNDGYYSF}  USZCNXEOJNCGYEOWEUPXEZGACGNFGLKNSACIGOIYCKXCJUCIUZC
    plain:  thee      t    he      e          h  e  t      e     e  e  e  he

    CIPHER: \textcolor{red}{FZCCNDGYYSF}  EUEKUZCSOC  \textcolor{red}{FZCCN}  CIACZEJNCSHFZEJZEGMXCYHCJUMGKUCY
    plain:  thee      t       he  e  thee   e  eh   e  th  h    e  e      e
	
\end{BVerbatim}

The prefix \verb|thee| suggests that \verb|F| probably does not correspond to 
\verb|t|. Consider the word {\color{red}\verb|wheel|} instead and use this to 
begin making some assertions and decrypt as shown below.

\begin{verbatim}
    i may not be able to grow flowers but my garden produces just as many dead leave sold 
    over shoes pieces of rope and bushels of dead grass as any bodys and today i bought a 
    wheelbarrow to help in clearing it up i have always loved and respected the 
    wheelbarrow it is the one wheeled vehicle of which i am perfect master
\end{verbatim}

\end{solution}


\newpage


%%=============================================================================
%%============================== PROBLEM C ====================================
%%=============================================================================

\subsection*{Problem C}

Decrypt the following Vigenère ciphertext using the Index of Coincidence 
method. [3 points]

% ciphertext from file
\VerbatimInput[xleftmargin=15pt]{scripts/hw01/c_cipher}

%=== SOLUTION ===
\begin{solution}

We begin by testing keyword periods and analyzing their corresponding Index of 
Coincidences.

% vscode snippet
\lstinputlisting[
	language = C, style = myCstyle,
	linerange = {92-131}
	]{hw01/ioc.c}

\newpage

The maximum of \verb|avgs| yields 10 but it's clear from the results below 
that 20 is also a possible period.

\begin{minipage}{0.25\linewidth}
	\begin{tabular}{ c c }
		\textbf{Period} & \textbf{Avg I.C.} \\
		1 & 0.0403894 \\
		2 & 0.0433027 \\
		\vdots & \vdots \\
		7 & 0.0404037 \\
		8 & 0.0434730 \\
		9 & 0.0400959 \\
		\rowcolor{yellow!75} 10 & 0.0636256 \\
		\vdots & \vdots \\
		\rowcolor{yellow!75} 20 & 0.0619492
	\end{tabular}
\end{minipage}
\begin{minipage}{0.7\linewidth}
	\begin{tikzpicture}
	\begin{axis}[
		title = {Average I.C. for Different Key Periods},
		xmin = 0, xmax = 21,
		xtick = data, tickwidth = 0pt,
		xticklabel style = {text height = 1.5ex},
		ymin = 0,
		ybar, bar width = 0.35cm,
		ymajorgrids = true, grid style = dashed,
		width = \textwidth, height = 0.38\textwidth
		]
		\addplot table {hw01/c_ic_values};   % values from file
	\end{axis}
	\end{tikzpicture}
\end{minipage}

\vspace{0.5cm}

We choose a period of 10 and try the 26 monoalphabetic ciphers to every 10th 
letter of the ciphertext to obtain the Chi-Square statistics. To do so, we 
need the most current frequency distribution according to Peter Norvig in 
2012 (\url{http://norvig.com/mayzner.html}).

% vscode snippet
\lstinputlisting[
	language = C, style = myCstyle,
	linerange = {47-57},
	numbersep = -10pt,     % workaround for removing 
	xleftmargin = -5pt     % tab from beginning
	]{hw01/ioc.c}

% vscode snippet
\lstinputlisting[
	language = C, style = myCstyle,
	linerange = {159-182},
	]{hw01/ioc.c}

\newpage

Notice that \verb|setFrequency| was defined in the previous page. The values 
of \verb|shift| and \verb|chi_vals| are shown below.

\begin{center}
\begin{tabular}{ c c c }
	\textbf{Key} & \textbf{Caesar Shift} & \textbf{Chi-Sq} \\
	a &	\verb|hfwrlmqdfuyryzzsyynjtjfwmstgynwrznkwlkxljkj...sgwjmfdnwbnqwjj| & 1139.65 \\
	b &	\verb|gevqklpcetxqxyyrxxmisievlrsfxmvqymjvkjwkiji...rfvilecmvampvii| & 3627.85 \\
	c &	\verb|fdupjkobdswpwxxqwwlhrhdukqrewlupxliujivjhih...qeuhkdbluzlouhh| & 876.493 \\
	d &	\verb|ectoijnacrvovwwpvvkgqgctjpqdvktowkhtihuighg...pdtgjcaktykntgg| & 4924.34 \\
	e &	\verb|dbsnhimzbqunuvvouujfpfbsiopcujsnvjgshgthfgf...ocsfibzjsxjmsff| & 826.637 \\
	\rowcolor{yellow!75} f &	
	\verb|carmghlyaptmtuunttieoearhnobtirmuifrgfsgefe...nbrehayirwilree| & 109.124 \\
	\vdots & \vdots & \vdots \\
	z &	\verb|igxsmnregvzszaatzzokukgxntuhzoxsaolxmlymklk...thxkngeoxcorxkk| & 4619.63
\end{tabular}
\end{center}

This suggest that the letter \verb|f| was used to encode every 10th plaintext 
letter. Hence, it is likely the first letter of the keyword. The full 
Chi-Square analysis yields the keyword \verb|fluxionate|.

% vscode snippet
\lstinputlisting[
	language = C, style = myCstyle,
	linerange = {201-216},
	]{hw01/ioc.c}

The full decrypt is shown below with spacing and punctuation applied.

% plaintext from file
\VerbatimInput[xleftmargin=15pt]{scripts/hw01/c_plain}

\end{solution}


\newpage


%%=============================================================================
%%============================== PROBLEM D ====================================
%%=============================================================================

\subsection*{Problem D}

Decrypt the following affine cipher. [3 points]

% ciphertext from file
\VerbatimInput[xleftmargin=15pt]{scripts/hw01/d_cipher}

%=== SOLUTION ===
\begin{solution}

Using scripts from previous problems, it's simple to generate our frequency 
distribution.

\begin{tikzpicture}
	\begin{axis}[
		title = {Frequency Distribution},
		xmin = 0, xmax = 1,                            % extra space at ends
		xtick = data, tickwidth = 0pt,
		xticklabel style = {text height = 1.5ex},	   % extra space between bars
		ymin = 0, ymax = 35,
		ymajorgrids = true, grid style = dashed,
		ybar, bar width = 0.5cm,
		width = \textwidth, height = 0.35\textwidth,
		symbolic x coords = {0,a,b,c,d,e,f,g,h,i,j,k,l,m,n,o,p,q,r,s,t,u,v,w,x,y,z,1}
		]
		\addplot table [header=false] {hw01/d_frequency};   % values from file
	\end{axis}
\end{tikzpicture}

From the frequency, it's likely that \verb|C| and \verb|B| are ciphers for 
\verb|e| and \verb|t| respectively. Assume this is true and solve the system 
below.
	$$\begin{bmatrix} 4 & 1 \\ 19 & 1 \end{bmatrix}
	\begin{bmatrix} a \\ b \end{bmatrix}
	\equiv
	\begin{bmatrix} 2 \\ 1 \end{bmatrix}
	\tmod{26} 
	\qquad \Rightarrow \qquad
	\begin{matrix} a = 19 \\ b = 4 \end{matrix}$$

Attempt decrypting by implementing 
$D([19,4],C) := 11(C-4) \equiv p \tmod{26}$ in the script below.

% vscode snippet
\lstinputlisting[
	language = C, style = myCstyle,
	linerange = {49-59},
	numbersep = -10pt,     % workaround for removing 
	xleftmargin = -5pt     % tab from beginning
	]{hw01/show_outputs.c}

By introducing spacing to the plaintext generated, we see that our guess 
worked since this yields the Canadian anthem!

\VerbatimInput[xleftmargin=15pt]{scripts/hw01/d_plain}

\end{solution}

\end{document}
