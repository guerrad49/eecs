\documentclass[../hw_sols.tex]{subfiles}
\setlist[description]{style = unboxed, leftmargin = 0.55cm}

\begin{document}

%%=============================================================================
%%=============================== PROBLEM 1 ===================================
%%=============================================================================

\subsection*{Problem 1}

Find the multiplicative inverse of each nonzero element in $\mathbb{Z}_7$.

%=== SOLUTION ===
\begin{solution}
The equivalences below show the multiplicative inverses.
\begin{align*}
	1^2 \; \equiv& \; 1 \tmod{7} \\
	2 \cdot 4 \; = \; 8 \; \equiv& \; 1 \tmod{7} \\
	3 \cdot 5 \; = \; 15 \; \equiv& \; 1 \tmod{7} \\
	6^2 = 36 \; \equiv& \; 1 \tmod{7}
\end{align*}
\end{solution}


\newpage


%%=============================================================================
%%=============================== PROBLEM 2 ===================================
%%=============================================================================

\subsection*{Problem 2}

Demonstrate whether each of these statements is true or false for polynomials 
over a field.

\begin{description}

%=== PART B ====
\item[(b)] The product of polynomials of degress $m$ and $n$ has degree $m+n$.

%=== SOLUTION ===
\begin{solution}
This statement is true. Take $f(x), g(x)$ to be polynomials over a field $F$. 
We may write $f(x) = ax^{m} + f_1(x)$ and $g(x) = bx^{n} + g_1(x)$ for 
$m,n \in \mathbb{Z}^+$ and $a,b \in \mathbb{Z}$, such that $\deg(f_1) < m$ and 
$\deg(g_1) < n$. If $m$ or $n$ equal 0, then $f_1$ or $g_1$ equal 0 
respectively. Then 
\begin{align*}
	f(x) \; \cdot& \; g(x) \\
	(ax^m + f_1(x))&(bx^n + g_1(x)) \\
	abx^{m+n} \; + \; ax^mg_1(x) \; +& \; bx^nf_1(x) \; + \; f_1(x)g_1(x).
\end{align*}
Notice that $\deg(ax^mg_1), \deg(bx^nf_1), \deg(f_1g_1) < m + n$. Since $F$ 
is a field, then $f \cdot g \in F$. Furthermore, $\deg(f \cdot g) = m + n$.
\end{solution}

%=== PART A ====
\item[(a)] The product of monic polynomials is monic.

%=== SOLUTION ===
\begin{solution}
This is merely a case of our previous proof. Let $a = b = 1$. This makes $f$ 
and $g$ monic, which consequently means that $f \cdot g$ is monic via the 
same steps in \textbf{(b)}.
\end{solution}

%=== PART C ====
\item[(c)] The sum of polynomials of degress $m$ and $n$ has degree max$[m,n]$.

%=== SOLUTION ===
\begin{solution}
This is certainly false. Every element in a field $F$ has an additive inverse 
so that $x \in F$ implies $-x \in F$. It's clear that 
$x + (-x) = 0 \neq \max\{\deg(x),\deg(-x)\}$.
\end{solution}

\end{description}


\newpage


%%=============================================================================
%%=============================== PROBLEM 3 ===================================
%%=============================================================================

\subsection*{Problem 3}

Determine the multiplicative inverse of $x^3 + x + 1$ in GF$(2^4)$ with 
$m(x) = x^4 + x + 1$.

%=== SOLUTION ===
\begin{solution}

It's simple enough to use the \textbf{Extended Euclidean Algorithm} to solve 
this problem. First we perform successive Euclidean divisions.
\begin{align*}
	x^4 + x + 1 \; =& \; x(x^3 + x + 1) + (x^2 + 1) \\
	x^3 + x + 1 \; =& \; x(x^2+ 1) + 1
\end{align*}

Then we express as a linear combination by back-solving.
\begin{align*}
	1 =& \; (x^3 + x + 1) - x(x^2 + 1) \\
	=& \; (x^3 + x + 1) - x[(x^4 + x+ 1) - x(x^3 + x + 1)] \\
	=& \; (1 + x^2)(x^3 + x + 1) - x(x^4 + x + 1)
\end{align*}

\noindent Thus, we have $(x^3+x+1)^{-1} = x^2 + 1$ in 
$\mathbb{Z}_2[x]/<x^4+x+1>$.

\end{solution}


\newpage


%%=============================================================================
%%=============================== PROBLEM 4 ===================================
%%=============================================================================

\subsection*{Problem 4}

Using Fermat's Theorem, find $3^{201} \tmod{11}$.

%=== SOLUTION ===
\begin{solution}
We know that since $\gcd(3,11) = 1$, then $3^{10} \equiv 1 \tmod{11}$ by 
\textbf{Fermat's Little Theorem}. Thus, we have 
	$$3^{201} = (3^{10})^{20} \cdot 3 
	\equiv 1^{20} \cdot 3 \equiv 3 \tmod{11}.$$
\end{solution}


\newpage


%%=============================================================================
%%=============================== PROBLEM 5 ===================================
%%=============================================================================

\subsection*{Problem 5}

Using Fermat's Theorem to find a number $x$ between 0 and 28 with $x^{85}$ 
congruent to 6 modulo 29. (You should not need to use any brute-force 
searching).

%=== SOLUTION ===
\begin{solution}
Here we have the computational task of solving $x^{85} \equiv 6 \tmod{29}$. 
Working in the field $\mathbb{Z}/29\mathbb{Z}$, call it $\mathbb{F}_{29}$, is 
great since $\gcd(x,29) = 1$ for all $x \in \mathbb{F}_{29}$. Then we state an 
alternate form of \textbf{FLT} that says $x^p \equiv x \tmod{p}$. This is easy 
to see via multiplication. Thus we have
	$$x^{85} = (x^{29})^2 \cdot x^{27} \equiv x^2 \cdot x^{27} 
	\equiv x^{29} \equiv x \equiv 6 \tmod{29}.$$
\end{solution}


\newpage


%%=============================================================================
%%=============================== PROBLEM 6 ===================================
%%=============================================================================

\subsection*{Problem 6}

Use Euler's Theorem to find a number $x$ between 0 and 28 with $x^{85}$ 
congruent to 6 modulo 35. (You should not need to use any brute-force 
searching).

%=== SOLUTION ===
\begin{solution}

This time we are required to solve $x^{85} \equiv 6 \tmod{35}$. Begin by 
computing $\varphi(35)$ using \textbf{Euler's product formula}.
	$$\varphi(35) 
	= 35 \prod_{p \mid 35} \left( 1 - \frac{1}{p} \right) 
	= 35 \left( 1 - \frac{1}{5} \right) \left( 1 - \frac{1}{7} \right) 
	= 24$$
Assume that $\gcd(x,35) = 1$. Then by \textbf{Euler's Theorem}, we have 
	$$(x^{24})^3 \cdot x^{13}
	 = (x^{\varphi(35)})^3 \cdot x^{13} 
	\equiv x^{13} \equiv 6 \tmod{35}.$$
While a brute-force approach wouldn't take as long now, it's not hard to solve 
using the \textbf{Chinese Remainder Theorem}. We've like to solve the 
simultaneous congruences 
$x^{13} \equiv 6 \tmod{5}$ and $x^{13} \equiv 6 \tmod{7}$. We'll
	$$x^{13} = (x^4)^3 \cdot x \equiv x \equiv 1 \tmod{5} \qquad 
	\text{ and } 
	\qquad x^{13} = (x^6)^2 \cdot x \equiv x \equiv 6 \tmod{7}.$$
Now $x = 5y + 1$ for $y \in \mathbb{Z}$ so that via substitution 
	$$5y + 1 \equiv 6 \tmod{7} 
	\quad \Rightarrow \quad 
	y \equiv 1 \tmod{7}.$$
Thus, $y = 7z + 1$ for $z \in \mathbb{Z}$ implies that 
$x = 5(7z+1) + 1 = 35z + 6$, which equivalently means $x \equiv 6 \tmod{35}$. 
Indeed, $\gcd(6,35) = 1$ and it's not hard to confirm with a computer that 
our result works (at least for a power as small as 85).

\end{solution}


\newpage


%%=============================================================================
%%=============================== PROBLEM 7 ===================================
%%=============================================================================

\subsection*{Problem 7}

The example used by Sun-Tsu to illustrate the CRT was
	$$x \equiv 2 \tmod{3} \; 
	; \; x \equiv 3 \tmod{5} \; 
	; \; x \equiv 2 \tmod{7}$$
Solve for $x$ using the Chinese Remainder Theorem.

%=== SOLUTION ===
\begin{solution}

Given the three equivalence relations, we may rewrite the first as $x = 3y + 2$ 
for $y \in \mathbb{Z}$. Note we've simply used the definition of equivalence. 
Next we substitute into the second to get 
	$$3y + 2 \equiv 3 \tmod{5} \quad 
	\Rightarrow \quad 3y \equiv 1 \tmod{5} \quad 
	\Rightarrow \quad y \equiv 2 \tmod{5}.$$
But then, we rewrite 
	$$x^{85} = (x^{29})^2 \cdot x^{27} \equiv x^2 \cdot x^{27} 
	\equiv x^{29} \equiv x \equiv 6 \tmod{29}$$
the result as $y = 5z + 2$ for $z \in \mathbb{Z}$, which implies that 
$x = 3(5z + 2) + 2 = 15z + 8$. Substitute into the third equivalence to get 
	$$15z + 8 \equiv 2 \tmod{7} 
	\quad \Rightarrow \quad 
	z \equiv 1 \tmod{7}.$$ 
One last time, $z = 7w + 1$ for $w \in \mathbb{Z}$. Thus, 
$x = 15(7w + 1) + 8 = 105w + 23$ means that $x \equiv 23 \tmod{105}$ by 
definition.

\end{solution}


\newpage


%%=============================================================================
%%=============================== PROBLEM 8 ===================================
%%=============================================================================

\subsection*{Problem 8}

Show that every integral domain has the \textit{cancellation property}: if 
$a \neq 0$ and $ax = ay$, then $x = y$.

%=== SOLUTION ===
\begin{solution}
Let $R$ be an integral domain with $a,x,y \in R$. Assume that $a \neq 0$ and 
$ax = ay$. It follows that $a(x - y) = 0$. But $S$ has no zero divisors, so 
either $a = 0$ or $x - y = 0$. Our assumption leads to the conclusion that 
$x - y = 0 \quad \Rightarrow \quad x = y$. Hence the 
\textit{cancellation property} holds in every integral domain.
\end{solution}


\newpage


%%=============================================================================
%%=============================== PROBLEM 9 ===================================
%%=============================================================================

\subsection*{Problem 9}

Show that every finite integral domain is a field.

%=== SOLUTION ===
\begin{solution}
Let $R$ be an integral domain. Assume that $|R| = m \in \mathbb{Z}^+$ and 
consider $R/\{0\}$. Now take $a,r_i \in R/\{0\}$ and look at the products 
$ar_i$. We claim that each product is different. The product is clearly a map 
$f: R/\{0\} \to R/\{0\}$. If $f$ is a bijection, we're done. Well, take 
$r_i, r_j \in R/\{0\}$ with $f(r_i) = f(r_j)$, then 
$ar_i = ar_j \quad \Rightarrow \quad r_i = r_j$ since $R$ has the cancellation 
property. This shows that $f$ is injective. However since $R/\{0\}$ is finite, 
$f$ is also surjective, hence bijective. This means that there must exist some 
$r_i$ such that $f(r_i) = ar_i = 1$. Notice that $1 \in R$ because it's an 
integral domain. Thus every nonzero element of $R$ is a unit which means $R$ 
is a field.
\end{solution}

\end{document}
