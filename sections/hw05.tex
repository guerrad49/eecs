\documentclass[../hw_sols.tex]{subfiles}
\setlist[description]{style = unboxed, leftmargin = 0.55cm}


\begin{document}

%%=============================================================================
%%=============================== PROBLEM 2.31 ================================
%%=============================================================================

\subsection*{Problem 2.31}

Show that, if $n$ is an odd composite integer, then the Miller-Rabin test will 
return inconclusive for $a = 1$ and $a = (n - 1)$.

%=== SOLUTION ===
\begin{solution}

Take a composite $n \in \mathbb{O}$. We can show the result by following each 
step of the algorithm.

\begin{itemize}

\item Let $a = 1$. Clearly $\gcd(a,n) = 1$ so we proceed. Factor 
$n-1 = 2^k \cdot m$ where $m \in \mathbb{O}$. Then $a_0 = a^m = 1 \tmod{n}$ 
and we are done. The test determines $a = 1$ as a prime.
	
\item Let $a = n-1$. Here $\gcd(a,n) = 1$ as well so we proceed. Factor as 
before so $n-1=2^k\cdot m$. Since $m$ is odd, write $m = 2l + 1$ for some 
integer $l$. We now compute $a_0 = (n-1)^m \tmod{n}$ where
	$$(n-1)^m = (n-1)^{2l+1} = (n-1)^{2l}(n-1) 
	\equiv 
	(-1)^{2l}(-1) \equiv -1 \tmod{n}.$$
Thus we have $a_0 = -1 \tmod{n}$ and the test determines $a = n-1$ as a prime.

\end{itemize}

\end{solution}


\newpage


%%=============================================================================
%%=============================== PROBLEM 2.32 ================================
%%=============================================================================

\subsection*{Problem 2.32}

If $n$ is composite and passes the Miller-Rabin test for the base $a$, then 
$n$ is called a strong pseudoprime to the base $a$. Show that 2047 is a strong 
pseudoprime to the base 2.

%=== SOLUTION ===
\begin{solution}
Begin with $a = 2$ and $n = 2047$ where $\gcd(2,2047) = 1$. Furthermore, 
$n-1 = 2046 = 2^1 \cdot 1023 = 2^k \cdot m$. Since $k=1$, we only need to 
compute $a_0$. Notice that $1023 = 99 \cdot 11$. Thus
	$$a_0 = 2^{1023} = (2^{11})^{99} = (2048)^{99} 
	\equiv 1^{99} 
	\equiv 1 \tmod{2047}$$
which means 2047 passes the \textbf{MRPT} and we say that it is a strong 
pseudoprime to the base 2.
\end{solution}


\newpage

%%=============================================================================
%%================================ PROBLEM 7.4 ================================
%%=============================================================================

\subsection*{Problem 7.4}

With the ECB mode, if there is an error in a block of the transmitted 
ciphertext, only the corresponding plaintext block is affected. However, in 
the CBC mode, this error propagates. For example, an error in the transmitted 
$C_1$ (Figure 7.4) obviously corrupts $P_1$ and $P_2$.

\begin{description}

%=== PART A ====
\item[(a)] Are any blocks beyond $P_2$ affected?

%=== SOLUTION ===
\begin{solution}
Blocks beyond $\mathbf{P_2}$ are not affected. In the encryption stage, 
$\mathbf{C_1}$ is no longer required after computing 
$\mathbf{C_1} \oplus \mathbf{P_2}$. Once $\mathbf{C_2}$ is generated, we 
require an error-less transmission so that $\mathbf{P_3}$ is decrypted 
correctly. We emphasize that $\mathbf{C_1}$ is NOT used in the decryption of 
$\mathbf{P_3}$.
\end{solution}

%=== PART B ====
\item[(b)] Suppose that there is a bit error in the source version of $P_1$. 
Through how many ciphertext blocks is this error propagated? What is the 
effect at the receiver?

%=== SOLUTION ===
\begin{solution}
A bit error in the source version of $\mathbf{P_1}$ is propagated through all 
the cipher blocks. This is due to the chained dependence that all cipher 
blocks $\mathbf{C_i}$ have on $\mathbf{P_1}$. Since the receiver now has 
incorrect cipher blocks $\mathbf{C_i}$, then the decryption of $\mathbf{P_i}$ 
will be incorrect as well.
\end{solution}

\end{description}


\newpage


%%=============================================================================
%%================================ PROBLEM 7.7 ================================
%%=============================================================================

\subsection*{Problem 7.7}

For the ECB, CBC, and CFB modes, the plaintext must be a sequence of one or 
more complete data blocks (or, for CFB mode, data segments). In other words, 
for these three modes, the total number of bits in the plaintext must be a 
positive multiple of the block (or segment) size. One common method of 
padding, if needed, consists of a 1 bit followed by as few zero bits, possibly 
none, as are necessary to complete the final block. It is considered good 
practice for the sender to pad every message, including messages in which the 
final message block is already complete. What is the motiva- tion for 
including a padding block when padding is not needed?

%=== SOLUTION ===
\begin{solution}
Padding every message could be a way of signaling the end of a message. More 
so, it can be a way of error checking. If there is tampering with the 
transmission channel, then the decryption will clearly have errors. The 
receiver will know there are clear errors if the padding structure is 
different from what he/she expects or has agreed upon with the sender.
\end{solution}


\newpage


%%=============================================================================
%%================================ PROBLEM 7.8 ================================
%%=============================================================================

\subsection*{Problem 7.8}

If a bit error occurs in the transmission of a ciphertext character in 8-bit 
CFB mode, how far does the error propagate?

%=== SOLUTION ===
\begin{solution}
Suppose we have a bit error in the transmission of $\mathbf{C_j}$, call the 
error $\mathbf{\tilde{C_j}}$. The receiver will compute 
$\mathbf{\tilde{C_j}} \oplus \text{MSB}_s(O_j) = \mathbf{\tilde{P_j}}$. Here 
$\mathbf{\tilde{P_j}}$ may not differ all that much from $\mathbf{P_j}$ since 
$\mathbf{C_j}$ was utilized at the end of decryption. However, there is a 
much bigger consequence in the $(j+1)^{\text{th}}$ step of decryption. Since 
$I_{j+1} = \text{LSB}_{b-s}(I_j) \Vert \mathbf{C_j}$, it is clear that 
$I_{j+1}$ depends on $\mathbf{C_j}$. Consequently, using 
$\mathbf{\tilde{C_j}}$ results in $\tilde{I}_{j+1}$. Furthermore, $O_{j+1}$ 
depends on $I_{j+1}$ and $\mathbf{P_{j+1}}$ depends on $O_{j+1}$. Thus, we can 
be sure that decryption of $\mathbf{C_{j+1}}$ is significantly altered. 
However, this is where error propagation stops since nothing else is dependent 
on $\mathbf{C_j}$.
\end{solution}


\newpage


%%=============================================================================
%%=============================== PROBLEM 7.10 ================================
%%=============================================================================

\subsection*{Problem 7.10}

In discussing the CTR mode, it was mentioned that if any plaintext block that 
is encrypted using a given counter value is known, then the output of the 
encryption function can be determined easily from the associated ciphertext 
block. Show the calculation.

%=== SOLUTION ===
\begin{solution}
Know that the CTR mode for encryption is defined as 
$\mathbf{C_j} = \mathbf{P_j} \oplus E(K,T_j)$. We are assuming that both 
$\mathbf{C_j}$ and $\mathbf{P_j}$ are known. It follows that 
	$$\mathbf{P_j} \oplus \mathbf{C_j} 
	= \mathbf{P_j} \oplus \mathbf{P_j} \oplus E(K,T_j) 
	\quad \Rightarrow \quad 
	\mathbf{P_j} \oplus \mathbf{C_j} = E(K,T_j).$$
This gives the output of the encryption function $E$. If we work over the 
decryption portion, we have
	$$\mathbf{C_j} \oplus \mathbf{P_j} 
	= \mathbf{C_j} \oplus \mathbf{C_j} \oplus E(K,T_j) 
	\quad \Rightarrow \quad 
	\mathbf{C_j} \oplus \mathbf{P_j} = E(K,T_j).$$
Notice that this is the exact same result as before since the $E$ is used both 
in encryption and decryption.
\end{solution}


\newpage


%%=============================================================================
%%=============================== PROBLEM 1 ===================================
%%=============================================================================

\subsection*{Problem 1}

Compute as much information as you can after each of the four transformations 
for one round of the AES encryption algorithm. The round key is 
\verb|07070606050504040303020201010000|. The input block looks like this at 
the beginning of the round:

\begin{center}
\begin{tabular}{| c | c | c | c |}
	\hline
	   &    &    & 7D \\ \hline
	7C &    &    &    \\ \hline
	   & D5 &    &    \\ \hline
	   &    & 54 &    \\ \hline
\end{tabular}
\end{center}

Most of the bytes have been left blank. Also, write the name of the 
appropriate AES transformation next to each block. (Use the tables below and 
recall that the irreducible polynomial used to generate the field GF($2^8$) 
in AES is $x^8 + x^4 + x^3 + x + 1$.)

% AES Figures
% TODO: Fix layout
\begin{figure}[h]
\begin{minipage}{0.5\textwidth}
	\VerbatimInput{inputs/hw05_sbox}
	\caption{S-box Table}
	\label{fig:sbox}
\end{minipage}
\hfill
\begin{minipage}{0.4\textwidth}
	\begin{verbatim}
        02 03 01 01
        01 02 03 01
        01 01 02 03
        03 01 01 02
	\end{verbatim}
	\caption{MixColumns Matrix}
	\label{fig:mixcols}
\end{minipage}
\end{figure}

%=== SOLUTION ===
\begin{solution}

\begin{description}

%=== PART A ====
\item[(a)] Begin by applying the S-box shown in figure \ref{fig:sbox} to the 
input block.
	\begin{center}
	\begin{tabular}{| c | c | c | c |}
		\hline
			&    &    & 7D \\ \hline
		7C &    &    &    \\ \hline
			& D5 &    &    \\ \hline
			&    & 54 &    \\ \hline
	\end{tabular}
	\qquad {\huge $\Rightarrow$} \qquad
	\begin{tabular}{| c | c | c | c |}
		\hline
			&    &    & FF \\ \hline
		10 &    &    &    \\ \hline
			& 03 &    &    \\ \hline
			&    & 20 &    \\ \hline
	\end{tabular} \quad \textbf{Substitute bytes}
	\end{center}

%=== PART B ====
\item[(b)] Next we shift the rows accordingly.
	\begin{center}
	\begin{tabular}{| c | c | c | c |}
		\hline
			&    &    & FF \\ \hline
		10 &    &    &    \\ \hline
			& 03 &    &    \\ \hline
			&    & 20 &    \\ \hline
	\end{tabular}
	\qquad {\huge $\Rightarrow$} \qquad
	\begin{tabular}{| c | c | c | c |}
		\hline
		\ \ \ \ & \ \ \ \ & \ \ \ \ & FF \\ \hline
		& & & 10 \\ \hline
		& & & 03 \\ \hline
		& & & 20 \\ \hline
	\end{tabular} \quad \textbf{Shift rows}
	\end{center}


\newpage


%=== PART C ====
\item[(c)] Now we mix the columns with the given AES MixColumns Matrix. 
However, since the first three columns are not known, we can only perform the 
transformation to the last column.
\begin{align*}
	\begin{bmatrix}
		02 & 03 & 01 & 01 \\
		01 & 02 & 03 & 01 \\
		01 & 01 & 02 & 03 \\
		03 & 01 & 01 & 02 
	\end{bmatrix} 
	\cdot 
	\begin{bmatrix}
		\text{FF} \\ 10 \\ 03 \\ 20
	\end{bmatrix} 
	= 
	\begin{bmatrix}
		s'_{0,3} \\ s'_{1,3} \\ s'_{2,3} \\ s'_{3,3}
	\end{bmatrix}
\end{align*}

Keep in mind that we're working modulo $f(x) = x^8 + x^4 + x^3 + x + 1$. The 
first multiplication step is shown below.
\begin{align*}
	s'_{0,3} \; 
	=& \; {\color{red} \text{FF}} \times 02 
	+ {\color{blue} 10} \times 03 
	+ {\color{green} 03} \times 01 
	+ {\color{violet} 20} \times 01 \\
	=& \; {\color{red} 11111111} \times 02 
	+ {\color{blue} 00010000} \times 03 
	+ {\color{green} 00000011} \times 01 
	+ {\color{violet} 00100000} \times 01 \\
	=& \; 111101101
\end{align*}

But then we work with the overflow part by XOR-ing by $f(x)$.
	$$s'_{0,3} = 111101101 \oplus 100011011 = 011110101$$
Then by discarding the leading zero, we have $s'_{0,3} = 1111 \; 0101 =$ F6. 
Similarly, we have
	\begin{center}
	$\begin{bmatrix}
		s'_{0,3} \\ s'_{1,3} \\ s'_{2,3} \\ s'_{3,3}
	\end{bmatrix} 
	= 
	\begin{bmatrix}
		\text{F6} \\ \text{FA} \\ 89 \\ 49
	\end{bmatrix}$
	\qquad {\huge $\Rightarrow$} \qquad
	\begin{tabular}{| c | c | c | c |}
		\hline
		\ \ \ \ & \ \ \ \ & \ \ \ \ & F6 \\ \hline
		& & & FA \\ \hline
		& & & 89 \\ \hline
		& & & 49 \\ \hline
	\end{tabular} \quad \textbf{Mix columns}
	\end{center}

%=== PART D ====
\item[(d)] The last step is to add the round key.
	\begin{center}
	\begin{tabular}{| c | c | c | c |}
		\hline
		07 & 05 & 03 & 01 \\ \hline
		07 & 05 & 03 & 01 \\ \hline
		06 & 04 & 02 & 00 \\ \hline
		06 & 04 & 02 & 00 \\ \hline
	\end{tabular}
	$\; \bigoplus \;$
	\begin{tabular}{| c | c | c | c |}
		\hline
		\ \ \ \ & \ \ \ \ & \ \ \ \ & F6 \\ \hline
		& & & FA \\ \hline
		& & & 89 \\ \hline
		& & & 49 \\ \hline
	\end{tabular}
	$\; = \;$
	\begin{tabular}{| c | c | c | c |}
		\hline
		\ \ \ \ & \ \ \ \ & \ \ \ \ & F7 \\ \hline
		& & & FB \\ \hline
		& & & 89 \\ \hline
		& & & 49 \\ \hline
	\end{tabular} \quad \textbf{Add round key}
	\end{center}

\end{description}

\end{solution}

\end{document}
