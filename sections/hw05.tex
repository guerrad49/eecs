\documentclass[../hw_sols.tex]{subfiles}
\setlist[description]{style = unboxed, leftmargin = 0.55cm}

\begin{document}

% =============================================================
% ======================== PROBLEM 2.31 =======================
% =============================================================

\subsection*{Problem 2.31}

Show that, if $n$ is an odd composite integer, then the Miller-Rabin test 
will return inconclusive for $a = 1$ and $a = (n - 1)$.

% ==== SOLUTION ====
\begin{solution}

For a composite $n \in \mathbb{O}$, suppose we find integers $k,q$ such that 
$n - 1 = 2^k \cdot q$ as stated on pg 69 of \cite{textbook}.

\begin{itemize}

\item If $a = 1$, then $a^q \tmod{n}$ is always 1 for any $q$ and $n$. The 
test concludes at line 3.

\item If $a = n-1$, then since $q$ is odd, 
\begin{align*}
	(n - 1)^q \tmod{n} \; 
	\equiv& \; (-1)^q \tmod{n} \\
	=& \; -1 \tmod{n} \\
	\equiv& \; n - 1.
\end{align*}
The test concludes at line 5 for the first iteration of $j$ (i.e. $j=0$).

\end{itemize}

In both cases, the Miller-Rabin test returns "inconclusive" for both $a = 1$ 
and $a = n - 1$.

\end{solution}


% =============================================================
% ======================== PROBLEM 2.32 =======================
% =============================================================

\subsection*{Problem 2.32}

If $n$ is composite and passes the Miller-Rabin test for the base $a$, then 
$n$ is called a strong pseudoprime to the base $a$. Show that 2047 is a 
strong pseudoprime to the base 2.

% ==== SOLUTION ====
\begin{solution}
Begin with the factorization $n - 1 = 2046 = 2^1 \cdot 1023$. The first check 
	\[ 2^{1023} = (2^{11})^{99} = 2048^{99} \equiv 1 \tmod{2047} \]
passes the \textbf{MRPT} and we say 2047 is a strong pseudoprime to the 
base 2.
\end{solution}


\newpage


% =============================================================
% ======================== PROBLEM 7.4 ========================
% =============================================================

\subsection*{Problem 7.4}

With the ECB mode, if there is an error in a block of the transmitted 
ciphertext, only the corresponding plaintext block is affected. However, in 
the CBC mode, this error propagates. For example, an error in the transmitted 
$C_1$ (Figure 7.4) obviously corrupts $P_1$ and $P_2$.

\begin{description}

% ==== PART A =====
\item[(a)] Are any blocks beyond $P_2$ affected?

% ==== SOLUTION ====
\begin{solution}
No. Know that 
$\mathbf{C_2} \oplus D(K, \mathbf{C_3}) = \mathbf{P_3}$	
is the first round not affected by an error in $\mathbf{C_1}$. If we assume 
there are no more transmission errors on $\mathbf{C_j}$ for $j > 2$, then no 
blocks beyond $\mathbf{P_2}$ are affected.
\end{solution}

% ==== PART B =====
\item[(b)] Suppose that there is a bit error in the source version of $P_1$. 
Through how many ciphertext blocks is this error propagated? What is the 
effect at the receiver?

% ==== SOLUTION ====
\begin{solution}
A bit error in the source version of $\mathbf{P_1}$ is propagated through all 
the cipher blocks. This is due to the chained dependence 
$\mathbf{C_{j-1}} \to \mathbf{P_j}$. The receiver should have erroneous 
versions of all $\mathbf{C_j}$ and thus fail to decrypt.
\end{solution}

\end{description}


% =============================================================
% ======================== PROBLEM 7.7 ========================
% =============================================================

\subsection*{Problem 7.7}

For the ECB, CBC, and CFB modes, the plaintext must be a sequence of one or 
more complete data blocks (or, for CFB mode, data segments). In other words, 
for these three modes, the total number of bits in the plaintext must be a 
positive multiple of the block (or segment) size. One common method of 
padding, if needed, consists of a 1 bit followed by as few zero bits, 
possibly none, as are necessary to complete the final block. It is considered 
good practice for the sender to pad every message, including messages in 
which the final message block is already complete. What is the motivation for 
including a padding block when padding is not needed?

% ==== SOLUTION ====
\begin{solution}
Padding every message could be a way of signaling the end of a message. More 
so, it can be a way of error checking. If there is tampering with the 
transmission channel, then the decryption will clearly have errors. The 
receiver will know there are clear errors if the padding structure is 
different from what he/she expects or has agreed upon with the sender.
\end{solution}


\newpage


% =============================================================
% ======================== PROBLEM 7.8 ========================
% =============================================================

\subsection*{Problem 7.8}

If a bit error occurs in the transmission of a ciphertext character in 8-bit 
CFB mode, how far does the error propagate?

% ==== SOLUTION ====
\begin{solution}
Suppose we have a bit error in the trasmission of $\mathbf{C_j}$. Certainly 
$\mathbf{P_j}$ is affected directly because 
$\mathbf{P_j} = \mathbf{C_j} \oplus \text{MSB}_s(O_j)$. However, notice 
the dependency chain $C_j \to I_{j+1} \to O_{j+1} \to P_{j+1}$. This means 
that 
\begin{itemize}
	\item $\mathbf{P_{j+1}}$ is altered and
	\item $I_{j+2}$ is altered because $I_{j+1} \to I_{j+2}$.
\end{itemize}

The second observation means all following $I_j$ are altered as well as their 
corresponding $\mathbf{P_j}$. Thus, the error propagates to all following 
rounds.
\end{solution}


% =============================================================
% ======================== PROBLEM 7.10 =======================
% =============================================================

\subsection*{Problem 7.10}

In discussing the CTR mode, it was mentioned that if any plaintext block that 
is encrypted using a given counter value is known, then the output of the 
encryption function can be determined easily from the associated ciphertext 
block. Show the calculation.

% ==== SOLUTION ====
\begin{solution}

Know that the CTR mode for encryption is defined as 
$\mathbf{C_j} = \mathbf{P_j} \oplus E(K,T_j)$. We are assuming that both 
$\mathbf{C_j}$ and $\mathbf{P_j}$ are known. It follows that 
	\[ \mathbf{P_j} \oplus \mathbf{C_j} 
	= \mathbf{P_j} \oplus \mathbf{P_j} \oplus E(K,T_j) 
	\quad \Rightarrow \quad 
	\mathbf{P_j} \oplus \mathbf{C_j} = E(K,T_j) \]

This gives the output of the encryption function $E$. If we work over the 
decryption portion, we have
	\[ \mathbf{C_j} \oplus \mathbf{P_j} 
	= \mathbf{C_j} \oplus \mathbf{C_j} \oplus E(K,T_j) 
	\quad \Rightarrow \quad 
	\mathbf{C_j} \oplus \mathbf{P_j} = E(K,T_j) \]

Notice that this is the exact same result as before since the $E$ is used both 
in encryption and decryption.

\end{solution}


\newpage


% =============================================================
% ========================= PROBLEM 1 =========================
% =============================================================

\subsection*{Problem 1}

Compute as much information as you can after each of the four transformations 
for one round of the AES encryption algorithm. The round key is 
\verb|07070606050504040303020201010000|. The input block looks like this at 
the beginning of the round:
\begin{center}
\begin{tabular}{| c | c | c | c |}
	\hline
	   &    &    & 7D \\ \hline
	7C &    &    &    \\ \hline
	   & D5 &    &    \\ \hline
	   &    & 54 &    \\ \hline
\end{tabular}
\end{center}

Most of the bytes have been left blank. Also, write the name of the 
appropriate AES transformation next to each block. (Use the tables below and 
recall that the irreducible polynomial used to generate the field GF($2^8$) 
in AES is $x^8 + x^4 + x^3 + x + 1$.)

% AES S-box.
\begin{figure}[h]
\centering
\begin{minipage}{0.5\textwidth}
	\VerbatimInput{verbatim/hw05/aes_sbox}
\end{minipage}
	\caption{S-box Table}
	\label{fig:sbox}
\end{figure}

\vspace{1cm}

% AES MixCols.
\begin{figure}[h]
\centering
\begin{minipage}{0.2\textwidth}
	\begin{verbatim}
    02 03 01 01
    01 02 03 01
    01 01 02 03
    03 01 01 02
	\end{verbatim}
\end{minipage}
	\caption{MixColumns Matrix}
	\label{fig:mixcols}
\end{figure}

\vspace{1cm}

% AES Irreducible.
\begin{figure}[h]
	\[ x^8 + x^4 + x^3 + x + 1 \]
	\caption{Irreducible Polynomial}
	\label{fig:poly}
\end{figure}


\newpage

% ==== SOLUTION ====
\begin{solution}

Begin by applying the \nameref{fig:sbox} to the input block and shifting the 
rows as shown below.

\begin{center}
\begin{minipage}[t]{0.3\textwidth}
\centering
\begin{tabular}{| c | c | c | c |}
	\hline
	   &    &    & 7D \\ \hline
	7C &    &    &    \\ \hline
	   & D5 &    &    \\ \hline
	   &    & 54 &    \\ \hline
\end{tabular}
\end{minipage}
{\Large $\Rightarrow$}
\begin{minipage}[t]{0.3\textwidth}
\centering
\begin{tabular}{| c | c | c | c |}
	\hline
	   &    &    & FF \\ \hline
	10 &    &    &    \\ \hline
	   & 03 &    &    \\ \hline
	   &    & 20 &    \\ \hline
\end{tabular} \\[4pt]
\textbf{Substitute bytes}
\end{minipage}
{\Large $\Rightarrow$}
\begin{minipage}[t]{0.3\textwidth}
\centering
\begin{tabular}{| *{3}{ p{10pt} |} c |}
	\hline
	& & & FF \\ \hline
	& & & 10 \\ \hline
	& & & 03 \\ \hline
	& & & 20 \\ \hline
\end{tabular} \\[4pt]
\textbf{Shift rows}
\end{minipage}
\end{center}

Next, apply \nameref{fig:mixcols} to the last column and handle overflow by 
XOR-ing by the \nameref{fig:poly}.
	\[
	\begin{bmatrix}
		02 & 03 & 01 & 01 \\
		01 & 02 & 03 & 01 \\
		01 & 01 & 02 & 03 \\
		03 & 01 & 01 & 02 
	\end{bmatrix} 
	\cdot 
	\begin{bmatrix}
		\text{FF} \\ 10 \\ 03 \\ 20
	\end{bmatrix} 
	= 
	\begin{bmatrix}
		\text{F6} \\
		\text{FA} \\
		\text{89} \\
		\text{49}
	\end{bmatrix}
	\]

Lastly, add the round key to get
\begin{center}
\begin{tabular}{| c | c | c | c |}
	\hline
	07 & 05 & 03 & 01 \\ \hline
	07 & 05 & 03 & 01 \\ \hline
	06 & 04 & 02 & 00 \\ \hline
	06 & 04 & 02 & 00 \\ \hline
\end{tabular}
$\; \bigoplus \;$
\begin{tabular}{| *{3}{ p{10pt} |} c |}
	\hline
	& & & F6 \\ \hline
	& & & FA \\ \hline
	& & & 89 \\ \hline
	& & & 49 \\ \hline
\end{tabular}
$\; = \;$
\begin{tabular}{| *{3}{ p{10pt} |} c |}
	\hline
	& & & F7 \\ \hline
	& & & FB \\ \hline
	& & & 89 \\ \hline
	& & & 49 \\ \hline
\end{tabular}
\end{center}

\end{solution}


\end{document}
