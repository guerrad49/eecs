\documentclass[../hw_sols.tex]{subfiles}
\setlist[description]{style = unboxed, leftmargin = 0.55cm}

\begin{document}

% =============================================================
% ========================= PROBLEM 1 =========================
% =============================================================

\subsection*{Problem 1}
\label{prob1}

Let $G$ be a group with group operation $\circ$, and $H \subseteq G$. Recall 
that
	\[ aH = \{a \circ h: h \in H\} \]
is a \textit{left coset} of $H$. Take any two elements $a, b \in G$. Show 
that if $aH \cap bH \neq \emptyset$, then $aH = bH$.

% ==== SOLUTION ====
\begin{solution}

Assume that $aH \cap bH \neq \emptyset$. Then there exists some 
$h_i,h_j \in H$ such that $a \circ h_i = b \circ h_j$. Then
\begin{align*}
	a \circ h_i \circ h_i^{-1} \; =& \; b \circ h_j \circ h_i^{-1} \\
	a \; =& \; b \circ \left( h_j \circ h_i^{-1} \right) \\
	a \circ h \; =& \; b \circ h_j \circ h_i^{-1} \circ h \\
	a \circ h \; =& \; b \circ h_k
\end{align*}

via group laws. This means that $a \circ h \in bH$ which implies that 
$aH \subseteq bH$. In the same manner, we can show that $b \circ h \in aH$ 
which implies that $bH \subseteq aH$. Thus $aH = bH$.

\end{solution}


\newpage


% =============================================================
% ========================= PROBLEM 2 =========================
% =============================================================

\subsection*{Problem 2}

\begin{description}

% ==== PART A ====
\item[a.] Compute $\phi(1525)$. (Recall that $\phi(n) = |Z^*_n|$ is Euler's 
totient function.)

% ==== SOLUTION ====
\begin{solution}
Know that 
	\[ \phi(n) = n \prod_{p | n} \left( 1 - \frac{1}{p} \right) \]
We factor $1525 = 5^2 \cdot 61$. Then 
	\[ \phi(1525) 
	= 1525 \left( 1 - \frac{1}{5} \right) \left( 1 - \frac{1}{61} \right) 
	= 1200 \]
\end{solution}

% ==== PART B ====
\item[b.] Use the Extended Euclidean Algorithm to compute $27^{-1} \tmod{41}$.

% ==== SOLUTION ====
\begin{solution}

To solve the congruence $27x \equiv 1 \tmod{41}$, start by performing a 
succession of Euclidan divisions.
\begin{align*}
	41 \; =& \; 1 \cdot 27 + 14 \\
	27 \; =& \; 1 \cdot 14 + 13 \\
	14 \; =& \; 1 \cdot 13 + 1
\end{align*}

Then we can substitute the successive remainders until we have expressed the 
two original numbers as a linear combination.
\begin{align*}
	1 \; 
	=& \; 14 \; - \; 13 \\
	=& \; (41 - 27) \; - \; (27 - 14) \\
	=& \; 41 \; - \; 2 \cdot 27 \; + \; 14 \\
	=& \; 41 \; - \; 2 \cdot 27 \; + \; (41 - 27) \\
	=& \; 2 \cdot 41 \; {\color{red} - \; 3} \cdot 27
\end{align*}

Thus $x = -3 \equiv 38 \tmod{41}$ is the multiplicative inverse of 27 in 
$\mathbb{Z}_{41}$.

\end{solution}

\end{description}


\newpage


% =============================================================
% ========================= PROBLEM 3 =========================
% =============================================================

\subsection*{Problem 3}

\begin{description}

% ==== PART A ====
\item[a.] Recall that an isomorphism from group $G$ to group $H$ is a 
one-to-one, onto function $f : G \to H$ such that 
$f(a \circ b) = f(a) \circ f(b)$ for all $a, b \in G$. We say that two groups 
are isomorphic if there is an isomorphism between them. Show that if $G$ is 
of order 4 and has an element of order 4, it is isomorphic to $\mathbb{Z}_4$.

% ==== SOLUTION ====
\begin{solution}

We wish to find a mapping $f: G \to H$ such that 
$(G, \circ) \simeq (\mathbb{Z}_4,+)$. Let ord$(g_1) = 4$ and assume that 
$g_1^k = g_k$ where we denote $g \circ g = g^2$. Since $G$ is a group, the 
operation table is uniquely determined as shown below on the left.

\begin{multicols}{2}
% ==== COLUMN 1 ====
\begin{center}
\begin{tabular}{ c | c | c | c | c }
	$\circ$ & $g_1$ & $g_2$ & $g_3$ & $g_4$ \\
	\hline
	  $g_1$ & $g_2$ & $g_3$ & $g_4$ & $g_1$ \\
	\hline
	  $g_2$ & $g_3$ & $g_4$ & $g_1$ & $g_2$ \\
	\hline
	  $g_3$ & $g_4$ & $g_1$ & $g_2$ & $g_3$ \\
	\hline
	  $g_4$ & $g_1$ & $g_2$ & $g_3$ & $g_4$
\end{tabular}
\end{center}

% ==== COLUMN 2 ====
\noindent
\begin{align*}
	f(g_i \circ g_j) \; 
	=& \; f(g_1^i \circ g_1^j) = \; f(g_1^{i+j}) \\
	=& \; f(g_{i+j}) \; = \; i + j \\
	=& \; f(g_i) + f(g_j)
\end{align*}
\end{multicols}

Hence we define $f(g_i) = i \tmod{4}$ which is one-to-one and onto. Lastly, 
$f$ is a homomorphism as shown above on the right. Note that all addition is 
done modulo 4. Thus $(G, \circ) \simeq (\mathbb{Z},+)$.

\end{solution}

% ==== PART B ====
\item[b.] Show that if $G$ is a group of order 4 and has no elements of order 
4, it is isomorphic to $\mathbb{Z}_2 \times \mathbb{Z}_2$.

% ==== SOLUTION ====
\begin{solution}

This time we want $(G,\circ) \simeq (\mathbb{Z}_2 \times \mathbb{Z}_2, +)$. 
We can determine the table for $G$ on the right. Here $g_4$ is the identity 
element again and we require $g_i^2 = g_4$, necessarily for all $i$, so all 
elements either have order 1 or 2. 

\begin{minipage}{0.65\linewidth}
We then define 
$\displaystyle f(g_i) = 
\left( 
	\left\lfloor \frac{i}{2} \right\rfloor \tmod{2}, \; i \tmod{2} 
\right)$.
Thus by explicitly defining $f$ such that
	\[ f(g_4) = (0,0) \; , \; 
	f(g_1) = (0,1) \; , \; 
	f(g_2) = (1,0) \; , \; 
	f(g_3) = (1,1) \]
we have shown that $(G,\circ) \simeq (\mathbb{Z}_2 \times \mathbb{Z}_2, +)$.
\end{minipage}
\quad\quad
\begin{minipage}{0.25\linewidth}
\begin{tabular}{ c | c | c | c | c }
	$\circ$ & $g_4$ & $g_1$ & $g_2$ & $g_3$ \\
	\hline
	  $g_4$ & $g_4$ & $g_1$ & $g_2$ & $g_3$ \\
	\hline
	  $g_1$ & $g_1$ & $g_4$ & $g_3$ & $g_2$ \\
	\hline
	  $g_2$ & $g_2$ & $g_3$ & $g_4$ & $g_1$ \\
	\hline
	  $g_3$ & $g_3$ & $g_2$ & $g_1$ & $g_4$
\end{tabular}
\end{minipage}

\end{solution}

\end{description}


\newpage


% =============================================================
% ========================= PROBLEM 4 =========================
% =============================================================

\subsection*{Problem 4}

\begin{description}

% ==== PART A ====
\item[a.] A subgroup $H$ of group $G$ is normal if $aH = Ha$ for all 
$a \in G$. Show that $H$ is a normal subgroup of $G$ if and only if every 
left coset of $H$ is also a right coset of $H$.

% ==== SOLUTION ====
\begin{solution}
\begin{description}
	\item[$(\Rightarrow)$] Assume $H$ is a normal subgroup of $G$. By 
	definition, we know that $aH = Ha$ for all $a \in G$. Since every left 
	coset $aH$ is the right coset $Ha$, we are done.

	\item[$(\Leftarrow)$] Assume that every left coset of $H$ is also a right 
	coset of $H$. Take $a \in G$ so that $aH$ is a left coset of $H$. By 
	assumption, there exists some $b \in G$ such that $aH = Hb$. Know that 
	$(H, \circ)$ is a group so it has the identity element, call it 
	$\textbf{e}$. It's easy to see that $a = a \circ \textbf{e} \in aH$ and 
	$a = \textbf{e} \circ a \in Ha$. However, $a \in Hb$ since $aH = Hb$. By 
	the result of \nameref{prob1}, this means that $Ha = Hb$. Thus 
	$aH = Ha$ and we conclude that $H$ is normal subgroup of $G$.
\end{description}
\end{solution}

% ==== PART B ====
\item[b.] The \textit{index} of a subgroup $H$ of $G$ is the number of left 
cosets of $H$ in $G$. Show that if $H$ is a subgroup of index 2 in $G$, then 
$H$ is a normal subgroup.

% ==== SOLUTION ====
\begin{solution}
Know that $H$ has index 2. This means that $G$ is partitioned by the two 
cosets of $H$, namely $H$ and $aH$ for some $a \in G$. More precisely, 
$H \sqcup aH = G$. We also know that the number of left cosets equals the 
number of right cosets. Using the same logic as before, we have 
$H \sqcup Hb = G$ for some $b \in G$. This gives us 
$H \sqcup aH = H \sqcup Hb \quad \Rightarrow \quad aH = Hb$. But this is just 
saying that every left coset is also a right coset! Note that $H$ is both a 
left and right coset of itself. Thus by the result of \textbf{a}, we conclude 
that $H$ is a normal subgroup of $G$.
\end{solution}

\end{description}


\newpage


% =============================================================
% ========================= PROBLEM 5 =========================
% =============================================================

\subsection*{Problem 5}

Recall from class that a group $G$ is cyclic if there is an element $a$ such 
every element of $G$ is a power of $a$. You may use the fact that 
$\mathbb{Z}^*_p$ is cyclic when $p$ is prime.

\begin{description}

% ==== PART A ====
\item[a.] Show that when $p \geq 3$ is prime, there are exactly two elements 
$a \in \mathbb{Z}^*_p$ such that $a^2 = 1$.

% ==== SOLUTION ====
\begin{solution}

It's easy to see that two solutions to $a^2 \equiv 1 \tmod{p}$, where 
$p \geq 3$, are $a_1 = 1$ and $a_2 = p - 1$. The first is trivially true while 
the second is easily shown below.
	\[ (p-1)^2 = p^2 - 2p + 1 \equiv 1 \tmod{p} \]
Suppose there exists some $\tilde{a} \neq a_1 \neq a_2$ that also solves the 
congruence. Then
\begin{align*}
	\tilde{a}^2 \; \equiv& \; 1 \tmod{p} \\
	\tilde{a}^2 - 1 \; \equiv& \; 0 \tmod{p} \\
	(\tilde{a} + 1)(\tilde{a} - 1) \; \equiv& \; 0 \tmod{p}.
\end{align*}
This means that $p \mid (\tilde{a} + 1)(\tilde{a} - 1)$ by definition. But if 
$p \mid (\tilde{a} + 1)$, we have
	\[ \tilde{a} + 1 \equiv 0 \tmod{p} 
	\quad \Rightarrow \quad 
	\tilde{a} \equiv -1 \equiv p - 1 \equiv a_2 \tmod{p} \]
On the other hand, if $p \mid (\tilde{a} - 1)$, we have 
$\tilde{a} \equiv a_1 \tmod{p}$. This contradicts the existence of any 
$\tilde{a}$ from our supposition. Thus the only solutions are 1 and $p-1$.

\end{solution}

% ==== PART B ====
\item[b.] Show that when $p \geq 3$ is prime,
	\[ (p - 1)! \; \equiv \; -1 \tmod{p} \]
Hint: $(p - 1)!$ is the product of all of the elements in the group 
$\mathbb{Z}^*_p$. What are the multiplicative inverses of these elements? 
Which are multiplicative inverses of themselves?

% ==== SOLUTION ====
\begin{solution}
Of course it's true that
	\[ (p-1)! \; = \; (p-1)(p-2) \cdots 3 \cdot 2 \cdot 1 \]
where the right hand side of the equality consists of all the elements in 
$\mathbb{Z}_p^*$. Moreover, all the elements are units! So for any element 
$a \in \mathbb{Z}_p^*$, have $a^{-1} \in \mathbb{Z}_p^*$. It's certainly 
possible that $a = a^{-1}$ for some $a$. However, our previous result 
indicates that this only occurs for two numbers, namely 1 and $p-1$. If we 
exclude those from product, we have
	\[ \underbrace{(p-2)(p-3) \cdots 3 \cdot 2}_{p-3 \text{ elements}} \]
Notice that $p-3$ is even since $p \geq 3$. Since this product is made up of 
units, we may rewrite
	\[ (p-2)(p-3) \cdots 3 \cdot 2 
	= \prod_{i=1}^{(p-3)/2} a_i \cdot a_i^{-1} \]
It follows that $(p-2)! \equiv 1 \tmod{p}$ and thus
	\[ (p-1)! = (p-1)(p-2)! \equiv p-1 \equiv -1 \tmod{p} \]
\end{solution}

\end{description}


\end{document}