\documentclass[../hw_sols.tex]{subfiles}
\setlist[description]{style = unboxed, leftmargin = 0.55cm}

\begin{document}

% =============================================================
% ========================= PROBLEM 1 =========================
% =============================================================

\subsection*{Problem 1}

Find the multiplicative inverse of each nonzero element in $\mathbb{Z}_7$.

% ==== SOLUTION ====
\begin{solution}
The equivalences below show the multiplicative inverses.
\begin{align*}
	1^2 \; \equiv& \; 1 \tmod{7} \\
	2 \cdot 4 \; = \; 8 \; \equiv& \; 1 \tmod{7} \\
	3 \cdot 5 \; = \; 15 \; \equiv& \; 1 \tmod{7} \\
	6^2 = 36 \; \equiv& \; 1 \tmod{7}
\end{align*}
\end{solution}


% =============================================================
% ========================= PROBLEM 2 =========================
% =============================================================

\subsection*{Problem 2}

Demonstrate whether each of these statements is true or false for polynomials 
over a field.

\begin{description}

% ==== PART B ====
\item[b.] The product of polynomials of degress $m$ and $n$ has degree $m+n$.

% ==== SOLUTION ====
\begin{solution}
This statement is true. Let 
\begin{center}
	$f(x) = ax^m + f_1(x)$ and $g(x) = bx^n + g_1(x)$
\end{center}
be polynomials over some field $F$ such that $\deg(f) = m$ and $\deg(g) = n$. 
Compute 
\begin{align*}
	f(x) \; \cdot \; g(x) \; 
	=& \; (ax^m + f_1(x)) (bx^n + g_1(x)) \\
	=& \; abx^{m+n} \; + \; ax^mg_1(x) \; + \; bx^nf_1(x) \; + \; f_1(x)g_1(x)
\end{align*}
Since $\deg(f_1) < m$ and $\deg(g_1) < n$, it follows that 
$\deg(x^mg_1)$, $\deg(x^nf_1)$, and $\deg(f_1g_1)$ are all less than $m + n$. 
Thus, $\deg(f \cdot g) = m + n$.
\end{solution}

% ==== PART A ====
\item[a.] The product of monic polynomials is monic.

% ==== SOLUTION ====
\begin{solution}
This is merely a case of our previous proof. Let $a = b = 1$. This makes $f$ 
and $g$ monic, which consequently means that $f \cdot g$ is monic via the 
same steps in \textbf{b.}.
\end{solution}

% ==== PART C ====
\item[c.] The sum of polynomials of degress $m$ and $n$ has degree max$[m,n]$.

% ==== SOLUTION ====
\begin{solution}
This is certainly false. Every element in a field $F$ has an additive inverse 
so that $f \in F$ implies $f^{-1} \in F$. It's clear that 
$f + f^{-1} = 0 \neq \max\{\deg(f),\deg(f^{-1})\}$.
\end{solution}

\end{description}


\newpage


% =============================================================
% ========================= PROBLEM 3 =========================
% =============================================================

\subsection*{Problem 3}

Determine the multiplicative inverse of $x^3 + x + 1$ in GF$(2^4)$ with 
$m(x) = x^4 + x + 1$.

% ==== SOLUTION ====
\begin{solution}

It's simple enough to use the \textbf{Extended Euclidean Algorithm} to solve 
this problem. First we perform successive Euclidean divisions.
\begin{align*}
	x^4 + x + 1 \; =& \; x(x^3 + x + 1) + (x^2 + 1) \\
	x^3 + x + 1 \; =& \; x(x^2+ 1) + 1
\end{align*}

Then we express as a linear combination by back-solving.
\begin{align*}
	1 =& \; (x^3 + x + 1) - x(x^2 + 1) \\
	=& \; (x^3 + x + 1) - x[(x^4 + x+ 1) - x(x^3 + x + 1)] \\
	=& \; (1 + x^2)(x^3 + x + 1) - x(x^4 + x + 1)
\end{align*}

\noindent Thus, we have $(x^3+x+1)^{-1} = x^2 + 1$ in 
$\mathbb{Z}_2[x]/<x^4+x+1>$.

\end{solution}


% =============================================================
% ========================= PROBLEM 4 =========================
% =============================================================

\subsection*{Problem 4}

Using Fermat's Theorem, find $3^{201} \tmod{11}$.

% ==== SOLUTION ====
\begin{solution}
We know that since $\gcd(3,11) = 1$, then $3^{10} \equiv 1 \tmod{11}$ by 
\textbf{Fermat's Little Theorem}. Thus, we have 
	\[ 3^{201} 
	= (3^{10})^{20} \cdot 3 
	\equiv 1^{20} \cdot 3 
	\equiv 3 \tmod{11} \]
\end{solution}


% =============================================================
% ========================= PROBLEM 5 =========================
% =============================================================

\subsection*{Problem 5}

Using Fermat's Theorem to find a number $x$ between 0 and 28 with $x^{85}$ 
congruent to 6 modulo 29. (You should not need to use any brute-force 
searching).

% ==== SOLUTION ====
\begin{solution}
Here we have the computational task of solving $x^{85} \equiv 6 \tmod{29}$. 
Since $\gcd(x,29) = 1$ for all $x \in \mathbb{Z}/29\mathbb{Z}$, we invoke 
the alternate form of \textbf{FLT} that says $x^p \equiv x \tmod{p}$. Thus 
	\[ x^{85} 
	= (x^{29})^2 \cdot x^{27} 
	\equiv x^2 \cdot x^{27} 
	\equiv x^{29} 
	\equiv x 
	\equiv 6 \tmod{29} \]
\end{solution}


\newpage


% =============================================================
% ========================= PROBLEM 6 =========================
% =============================================================

\subsection*{Problem 6}

Use Euler's Theorem to find a number $x$ between 0 and 28 with $x^{85}$ 
congruent to 6 modulo 35. (You should not need to use any brute-force 
searching).

% ==== SOLUTION ====
\begin{solution}

This time we are required to solve $x^{85} \equiv 6 \tmod{35}$. Using the 
\textbf{Chinese Remainder Theorem}, we can write the system of congruences
\begin{multicols}{2}
\noindent
\begin{align*}
	x^{85} \equiv 6 &\equiv 1 \tmod{5} \\
	   (x^4)^{21} x &\equiv 1 \tmod{5} \\
	              x &\equiv 1 \tmod{5}
\end{align*}
\noindent
\begin{align*}
	      x^{85} &\equiv 6 \tmod{7} \\
	(x^6)^{14} x &\equiv 6 \tmod{7} \\
	           x &\equiv 6 \tmod{7}
\end{align*}
\end{multicols}

Rewrite the first congruence as $x = 5t + 1$ for $t \in \mathbb{Z}$. Then 
substitute into the second congruence so that
	\[ (5t + 1) \equiv 6 \tmod{7} 
	\quad \Rightarrow \quad 
	t \equiv 1 \tmod{7} 
	\quad \Rightarrow \quad 
	t = 7k + 1 \quad \text{for } k \in \mathbb{Z} \]

Lastly, back substitute to get
	\[ x = 5(7k + 1) + 1 = 35k + 6 
	\quad \Rightarrow \quad 
	x \equiv 6 \tmod{35} \]

Since $x$ must be between 0 and 28, $x$ must equal 6.

\end{solution}


% =============================================================
% ========================= PROBLEM 7 =========================
% =============================================================

\subsection*{Problem 7}

The example used by Sun-Tsu to illustrate the CRT was
	\[ x \equiv 2 \tmod{3} \; 
	; \; x \equiv 3 \tmod{5} \; 
	; \; x \equiv 2 \tmod{7} \]
Solve for $x$ using the Chinese Remainder Theorem.

% ==== SOLUTION ====
\begin{solution}

Rewrite the first congruence as
\begin{equation}
	x = 3t + 2 \quad \text{for } t \in \mathbb{Z}
	\label{eq:first}
\end{equation}

Substitute \eqref{eq:first} into the second congruence so that 
	\[ 3t + 2 \equiv 3 \tmod{5} 
	\quad \Rightarrow \quad 
	t \equiv 2 \tmod{5} 
	\quad \Rightarrow \quad 
	t = 5k + 2 \quad \text{for } k \in \mathbb{Z} \]

Then plugging back into \eqref{eq:first} yields 
\begin{equation}
	x = 15k + 8
	\label{eq:second}
\end{equation}

Substitute \eqref{eq:second} into the third congruence so that
	\[ 15k + 8 \equiv 2 \tmod{7} 
	\quad \Rightarrow \quad 
	k \equiv 1 \tmod{7} 
	\quad \Rightarrow \quad 
	k = 7w + 1 \quad \text{ for } w \in \mathbb{Z} \]

One final substitution into \eqref{eq:second} yields
	\[ x = 15(7w + 1) + 8 = 105w + 23 
	\quad \Rightarrow \quad 
	x \equiv 23 \tmod{105} \]

Thus, one solution for $x$ can be 23.

\end{solution}


\newpage


% =============================================================
% ========================= PROBLEM 8 =========================
% =============================================================

\subsection*{Problem 8}
\label{prob8}

Show that every integral domain has the \textit{cancellation property}: if 
$a \neq 0$ and $ax = ay$, then $x = y$.

% ==== SOLUTION ====
\begin{solution}
Take $a,x,y \in R$ an integral domain. Assume that $a \neq 0$ and $ax = ay$. 
It follows that $a(x - y) = 0$. But since $a \neq 0$, then $x - y = 0$ or 
rather $x = y$. Hence the \textit{cancellation property} holds in every 
integral domain.
\end{solution}


% =============================================================
% ========================= PROBLEM 9 =========================
% =============================================================

\subsection*{Problem 9}

Show that every finite integral domain is a field.

% ==== SOLUTION ====
\begin{solution}
Let $R$ be an integral domain. Let $a \in R$ such that $a \neq 0$ and 
consider the map $f: R \to R$ where $f(x) = ax$. By \nameref{prob8}, we know 
that $f$ is injective since $ax = ay \quad \Rightarrow \quad x = y$. Now, 
since $R$ is finite and injective, then $f$ is also surjective. This means 
that there exist some $b \in R$ such that $ab = 1$ for any $a \in R$. Thus, 
every nonzero element of $R$ has a multiplicative inverse and $R$ is 
consequently a field.
\end{solution}


\end{document}