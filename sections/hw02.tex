\documentclass[../hw_sols.tex]{subfiles}
\setlist[description]{style = unboxed, leftmargin = 0.55cm}


\begin{document}

%%=============================================================================
%%============================== PROBLEM 4.2 ==================================
%%=============================================================================
	
\subsection*{Problem 4.2}

Consider a Feistel cipher composed of sixteen rounds with a block length of 
128 bits and a key length of 128 bits. Suppose that, for a given $k$, the key 
scheduling algorithm determines values for the first eight round keys, 
$k_1, k_2, \dots k_8$, and then sets
	$$k_9 = k_8, k_{10} = k_7, k_{11} = k_6, \dots, k_{16} = k_1$$
Suppose you have a ciphertext $c$. Explain how, with access to an encryption 
oracle, you can decrypt $c$ and determine $m$ using just a single oracle query. 
This shows that such a cipher is vulnerable to a chosen plaintext attack. (An 
encryption oracle can be thought of as a device that, when given a plaintext, 
returns the corresponding ciphertext. The internal details of the device are 
not known to you and you cannot break open the device. You can only gain 
information from the oracle by making queries to it and observing its 
responses.)

%=== SOLUTION ===
\begin{solution}
If we are given a subkey sequence 
	$$k_1 k_2 \dots k_8 k_8 \dots k_2 k_1,$$ 
the Feistel encryption and decryption algorithms become identical. This is 
due to the symmetry of the subkey sequence. Thus, with access to an 
encryption oracle and knowledge of a ciphertext $\mathbf{c}$, we simply query 
$E(\mathbf{c}) = D(\mathbf{c}) = \mathbf{m}$.
\end{solution}


\newpage


%%=============================================================================
%%============================== PROBLEM 4.5 ==================================
%%=============================================================================

\subsection*{Problem 4.5}

For any block cipher, the fact that it is a nonlinear function is crucial to 
its security. To see this, suppose that we have a linear block cipher $EL$ 
that encrypts 256-bit blocks of plaintext into 256-bit blocks of ciphertext. 
Let $EL(k, m)$ denote the encryption of a 256-bit message $m$ under a key $k$ 
(the actual bit length of $k$ is irrelevant). Thus,
	$$EL(k,[m_1 \oplus m_2]) = 
	EL(k,m_1) \oplus EL(k,m_2) \text{ for all 128-bit patterns } m_1,m_2.$$
Describe how, with 256 chosen ciphertexts, an adversary can decrypt any 
ciphertext without knowledge of the secret key $k$. (A “chosen ciphertext” 
means that an adversary has the ability to choose a ciphertext and then obtain 
its decryption. Here, you have 256 plaintext/ciphertext pairs to work with and 
you have the ability to choose the value of the ciphertexts.)

%=== SOLUTION ===
\begin{solution}
Consider the set of ciphertexts $\mathcal{C}$. Let $c_i \in \{0,1\}^{128}$ 
with $1 \leq i \leq 128$ be the chosen ciphertexts.  Each $c_i$ has one in 
the $i^{th}$ position, zeros elsewhere and a corresponding plaintext $m_i$. So
	$$EL(k,\mathbf{m}) 
	= EL \left( k, \bigoplus_{i=1}^n m_i \right) 
	\stackrel{linearity}{=\joinrel=\joinrel=} \bigoplus_{i=1}^n EL(k,m_i) 
	= \bigoplus_{i=1}^n c_i 
	= \mathbf{c}$$
where $1 \leq n \leq 128$ describes encryption. More importantly, 
$\displaystyle \bigoplus_{i=1}^n m_i$ corresponds to $\mathbf{c}$ and the 
adversary can easily compute $\mathbf{m}$. Note that 
$EL(k,\mathbf{0}) = \mathbf{0}$.
\end{solution}


\newpage


%%=============================================================================
%%============================== PROBLEM 4.6 ==================================
%%=============================================================================

\subsection*{Problem 4.6}

Suppose the DES F function mapped every 32-bit input R, regardless of the value 
of the input K, to \textbf{a} and \textbf{b}.
\begin{enumerate}
	\item What function would DES then compute?
	\item What would the decryption look like?
\end{enumerate}

\begin{description}

%=== PART A ====
\item[a.] 32-bit string of zero

%=== SOLUTION ===
\begin{solution}

For message $\mathbf{M}$, we have IP$(\mathbf{M}) = L_0 \Vert R_0$ and observe 
one round of DES encryption below [left].

\begin{multicols}{3}
	\hfill   % flush right
	\includestandalone[scale=0.75]{figures/simple_des_enc}

	\begin{center}
	\begin{tabular}{ c | c | c }
		\textbf{Round} & $L_i$ & $R_i$ \\[2pt]   % extra row height
		\hline
		IP & $L_0$ & $R_0$ \\[5pt]
		 1 & $R_0$ & $L_0$ \\[5pt]
		 2 & $L_0$ & $R_0$ \\[5pt]
		16 & $L_0$ & $R_0$ \\[5pt]
	\end{tabular}
	\end{center}

	\includestandalone[scale=0.75]{figures/simple_des_dec}
\end{multicols}

\vspace{-0.25cm}
The DES function described is $F(R) = \mathbf{0}$. It's clear from the table 
above [middle] that $\{L,R\}_0 = \{L,R\}_2 = \{L,R\}_{16}$. To finish 
encrypting, compute $\text{IP}^{-1}(R_0 \Vert L_0) = \mathbf{C}$.

For decryption, perform at 32-bit swap on IP($\mathbf{C}$) which yields 
$L_0 \Vert R_0$. One round of decryption, shown above [right], is simply a 
swap. As with encryption, 16 rounds yields a circular result. To finish, 
compute $\text{IP}^{-1}\big( L_0 \Vert R_0 \big) = \textbf{M}$.

\end{solution}


%=== PART B ====
\item[b.] R

%=== SOLUTION ===
\begin{solution}

The tables below, show encryption on the left and decryption on the right.

\begin{multicols}{2}
	\begin{center}
	\begin{tabular}{ c | c | c | c }
		\textbf{Round} & $L_i$ & $R_i$ & Simplified \\[2pt]   % extra row height
		\hline
		IP & $L_0$ & $R_0$ \\[5pt]
		 1 & $R_0$ & $L_0 \oplus R_0$ & $R_1$ \\[5pt]
		 2 & $R_1$ & $R_0 \oplus R_1$ & $L_0$ \\[5pt]
		 3 & $L_0$ & $R_1 \oplus L_0$ & $R_0$ \\[5pt]
		16 & $R_0$ & $L_0 \oplus R_0$ & $R_1$
	\end{tabular}
	\end{center}

	\begin{center}
	\begin{tabular}{ c | c | c | c }
		\textbf{Round} & $L_i$ & Simplified & $R_i$ \\[2pt]   % extra row height
		\hline
		16 &               $R_0$ &          & $R_1$    \\[5pt]
		15 &    $R_0 \oplus R_1$ &    $L_0$ & $R_0$    \\[5pt]
		14 &    $L_0 \oplus R_0$ & $R_{14}$ & $L_0$    \\[5pt]
		13 & $R_{14} \oplus L_0$ &    $R_0$ & $R_{14}$ \\[5pt]
		12 & $R_0 \oplus R_{14}$ &    $L_0$ & $R_0$
	\end{tabular}
	\end{center}
\end{multicols}

For encryption, have IP($\mathbf{M}) = L_0 \Vert R_0$ with $F(X) = X$ i.e. the 
identify function. Notice,
	$$R_{i+1} = L_i \oplus F(R_i) = L_i \oplus R_i.$$
Then $\{L,R\}_0 = \{L,R\}_3 = \{L,R\}_{15}$. To finish encrypting, compute
$\text{IP}^{-1}(R_1 \Vert R_0) = \mathbf{C}$.

To decrypt, certainly $F = F^{-1}$. Begin with 
$\text{IP}(\mathbf{C}) = R_1 \Vert R_0$ followed by 32-bit swap which yiels 
$R_0 \Vert R_1$. In this case,
	$$L_{i-1} = R_{i-1} \oplus R_i.$$
Then $\{L,R\}_{15} = \{L,R\}_{12} = \{L,R\}_{0}$. To finish decrypting, compute
$\text{IP}^{-1}\big( L_0 \Vert R_0 \big) = \textbf{M}$.

\end{solution}

\end{description}


\newpage


%%=============================================================================
%%============================== PROBLEM 4.11 =================================
%%=============================================================================

\subsection*{Problem 4.11}

This problem provides a numerical example of encryption using a one-round 
version of DES. We start with the same bit pattern for the key K and the 
plaintext, namely:

\begin{center}
\begin{tabular} { l c l }
	\textbf{Hexadecimal notation:} & \qquad & 0 1 2 3 4 5 6 7 8 9 A B C D E F \\
	\textbf{Binary Notation:} & \qquad & 0000 0001 0010 0011 0100 0101 0110 0111 \\
	& \qquad & 1000 1001 1010 1011 1100 1101 1110 1111
\end{tabular}
\end{center}

\begin{description}

%=== PART A ====
\item[a.] Derive $K_1$, the first-round subkey.

%=== SOLUTION ====
\begin{solution}

The code below computes $K_1$ according to Figure 4.5 in the text.

% vscode snippet
\lstinputlisting[
	language = C, style = myCstyle,
	linerange = {5-6,16-33}
	]{hw02/des.c}

The output of our code is the 48-bit key
	$$\mathbf{K_1} 
	= \verb|0x0b02679b49a5|
	= \verb|000010 110000 001001 100111 100110 110100 100110 100101|$$

\end{solution}


%=== PART B ====
\item[b.] Derive $L_0$, $R_0$.

%=== SOLUTION ====
\begin{solution}

Since $\mathbf{M} = \mathbf{K}$, have:

% vscode snippet
\lstinputlisting[
	language = C, style = myCstyle,
	linerange = {37-40},
	numbersep = -10pt,     % workaround for removing 
	xleftmargin = -5pt     % tab from beginning
	]{hw02/des.c}

\begin{center}
	$\mathbf{L_0}$ = \verb|0xcc00ccff| = \verb|11001100 00000000 11001100 11111111| \\
	$\mathbf{R_0}$ = \verb|0xf0aaf0aa| = \verb|11110000 10101010 11110000 10101010|
\end{center}

\end{solution}


%=== PART C ====
\item[c.] Expand $R_0$ to get $E[R_0]$, where $E[\cdot]$ is the expansion 
function of Table S.1.

%=== SOLUTION ====
\begin{solution}
The result of \verb|applyTransform("expansion", r_zero, 32)| is
\begin{center}
	$\mathbf{E[R_0]}$ 
	= \verb|0x7a15557a1555|
	= \verb|01111010 00010101 01010101 01111010 00010101 01010101|
\end{center}
\end{solution}


\newpage


%=== PART D ====
\item[d.] Calculate $A = E[R_0] \oplus K_1$.

%=== SOLUTION ====
\begin{solution}
	$\mathbf{A}$ 
	= \verb|0x711732e15cf0| 
	= \verb|01110001 00010111 00110010 11100001 01011100 11110000|
\end{solution}


%=== PART E ====
\item[e.] Group the 48-bit result of (d) into sets of 6 bits and evaluate the 
corresponding S-box substitutions.

%=== SOLUTION ====
\begin{solution}

Regroup as follows.
	\begin{center}
	\begin{tabular}{ c c c c }
		$\mathbf{B_1}$ = \verb|011100| & $\mathbf{B_2}$ = \verb|010001| &
		$\mathbf{B_3}$ = \verb|011100| & $\mathbf{B_4}$ = \verb|110010| \\
		$\mathbf{B_5}$ = \verb|111000| & $\mathbf{B_6}$ = \verb|010101| &
		$\mathbf{B_7}$ = \verb|110011| & $\mathbf{B_8}$ = \verb|110000|
	\end{tabular}
	\end{center}

Then apply S-box substitutions accordingly.

% vscode snippet
\lstinputlisting[
	language = C, style = myCstyle,
	linerange = {55-66},
	numbersep = -10pt,     % workaround for removing 
	xleftmargin = -5pt     % tab from beginning
	]{hw02/des.c}

This gives us
	\begin{center}
	\begin{tabular}{ c c c c }
		$\mathbf{S(B_1)}$ = \verb|0000| & $\mathbf{S(B_2)}$ = \verb|1100| &  
		$\mathbf{S(B_3)}$ = \verb|0010| & $\mathbf{S(B_4)}$ = \verb|0001| \\
		$\mathbf{S(B_5)}$ = \verb|0110| & $\mathbf{S(B_6)}$ = \verb|1101| &
		$\mathbf{S(B_7)}$ = \verb|0101| & $\mathbf{S(B_8)}$ = \verb|0000|.
	\end{tabular}
	\end{center}

\end{solution}


%=== PART F ====
\item[f.] Concatenate the results of (e) to get a 32-bit result, $B$.

%=== SOLUTION ====
\begin{solution}

Using the code below, we get: 
$\mathbf{B}$ = \verb|0x0c216d50| = \verb|00001100001000010110110101010000|.

% vscode snippet
\lstinputlisting[
	language = C, style = myCstyle,
	linerange = {70-74},
	numbersep = -10pt,     % workaround for removing 
	xleftmargin = -5pt     % tab from beginning
	]{hw02/des.c}

\end{solution}


%=== PART G ====
\item[g.] Apply the permutation to get $P(B)$.

%=== SOLUTION ====
\begin{solution}
The result of \verb|applyTransform("permutation", B, 32)| is
\begin{center}
	$\mathbf{P(B)}$ 
	= \verb|0x921c209c|
	= \verb|10010010000111000010000010011100|
\end{center}
\end{solution}


%=== PART H ====
\item[h.] Calculate $R1 = P(B) \oplus L_0$.

%=== SOLUTION ====
\begin{solution}
$\mathbf{R_1} 
= \mathbf{P(B)} \oplus \mathbf{L_0}$
= \verb|0x5e1cec63|
= \verb|01011110000111001110110001100011|.
\end{solution}


\newpage


%=== PART I ====
\item[i.] Write down the ciphertext.

%=== SOLUTION ====
\begin{solution}

Perform a 32-bit swap on $\mathbf{L_1 \Vert R_1}$ and apply the inverse 
permutation as shown below.

% vscode snippet
\lstinputlisting[
	language = C, style = myCstyle,
	linerange = {89-91},
	numbersep = -10pt,     % workaround for removing 
	xleftmargin = -5pt     % tab from beginning
	]{hw02/des.c}

Thus, the ciphertext is
\begin{center}
	\verb|0000 0001 0110 0011 0101 0100 0111 0110 1101 1000 1010 1111 1100 1101 1010 1110| \\
	\verb|0x 0 1 6 3 5 4 7 6 D 8 A F C D A E|
\end{center}

\end{solution}

\end{description}


\vspace{0.5cm}


%=== NOTES ===
\begin{solution}
Finally, we made use of the three functions defined below.

% vscode snippet
\lstinputlisting[
	language = C, style = myCstyle,
	linerange = {100-135}
	]{hw02/des.c}
\end{solution}


\newpage


%%=============================================================================
%%============================== PROBLEM 4.14 =================================
%%=============================================================================

\subsection*{Problem 4.14}

\begin{description}

%=== PART A ====
\item[a.] Let $X'$ be the bitwise complement of $X$. Prove that if the 
complement of the plaintext block is taken and the complement of an encryption 
key is taken, then the result of DES encryption with these values is the 
complement of the original ciphertext. That is,
\begin{align*}
	\text{If} \qquad Y \quad &= \quad E(K, X) \\
	\text{Then} \qquad Y' \quad &= \quad E(K', X')
\end{align*}

%=== SOLUTION ====
\begin{solution}

To begin, notice that 
	$$X = L_0 \Vert X_0 \quad \Rightarrow \quad X' = L'_0 \Vert R'_0.$$
On the other hand, let $P$ be a bitwise permutation such that $P(b_i) = b_j$ 
where $b_i, b_j$ are the $i^{\text{th}}$ and $j^{\text{th}}$ bits respectively. 
It follows that $P(b'_i) = b'_j$. Then
	$$PC1(K) = C_0 \Vert D_0 \quad 
	\Rightarrow 
	\quad PC1(K') = C'_0 \Vert D'_0.$$
Likewise, applying a left circular shift and $PC2$ yield the expected results. 
See diagram below for reference.
\begin{center}
	\includestandalone[scale=0.75]{figures/full_des}
\end{center}

Now we address the more involved portion. At the first XOR, we have 
$E(R'_0) \oplus K'_1$. Below [left], we show that 
$A = E(R) \oplus K = E(R') \oplus K'$.
	\begin{center}
	\begin{tabular}{ *{6}{ c } }
		$E$ & $K$ & $E \oplus K$ & $E'$ & $K'$ & $E' \oplus K'$ \\
		\hline
		0 & 0 & \cellcolor{yellow!75} 0 & 1 & 1 & \cellcolor{yellow!75} 0 \\
		0 & 1 & \cellcolor{yellow!75} 1 & 1 & 0 & \cellcolor{yellow!75} 1 \\
		1 & 0 & \cellcolor{yellow!75} 1 & 0 & 1 & \cellcolor{yellow!75} 1 \\
		1 & 1 & \cellcolor{yellow!75} 0 & 0 & 0 & \cellcolor{yellow!75} 0 
	\end{tabular}
	\hspace{1cm}
	\begin{tabular}{ *{6}{ c } }
		$L$ & $P$ & $L \oplus P$ & $(L \oplus P)'$ & $L'$ & $L' \oplus P$ \\
		\hline
		0 & 0 & 0 & \cellcolor{yellow!75} 1 & 1 & \cellcolor{yellow!75} 1 \\
		0 & 1 & 1 & \cellcolor{yellow!75} 0 & 1 & \cellcolor{yellow!75} 0 \\
		1 & 0 & 1 & \cellcolor{yellow!75} 0 & 0 & \cellcolor{yellow!75} 0 \\
		1 & 1 & 0 & \cellcolor{yellow!75} 1 & 0 & \cellcolor{yellow!75} 1 
	\end{tabular}
	\end{center}
Applying the S-boxes then yields the same result $B$ and for the second XOR, 
we see above [right] that \newline $L' \oplus P = (L \oplus P)' = R'$.

It follows that Round 16 would yield $L'_{16}$ and $R'_{16}$. Finally, since 
$IP^{-1}$ is a permutation as well
	$$IP^{-1}(R'_{16} \Vert L'_{16}) = (IP^{-1}(R_{16} \Vert L_{16}))' = Y'$$
Thus, proving if $Y = E(K,X)$, then $Y' = E(K',X')$.

\end{solution}
	
%=== PART B ====
\item[b.] It has been said that a brute-force attack on DES requires searching 
a key space of $2^{56}$ keys. Does the result of part (a) change that?

%=== SOLUTION ====
\begin{solution}
In any set of $2^n$ bit keys, half of them are complements of the other half. 
This makes it computationally inexpensive to find $Y'$ if a brute-force attack 
has already been performed to find $Y$. Hence, the true key space is actually 
$2^{56}/2 = 2^{55}$, which is still large.
\end{solution}

\end{description}


\end{document}