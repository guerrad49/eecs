\documentclass[../hw_sols.tex]{subfiles}
\setlist[description]{style = unboxed, leftmargin = 0.55cm}

\begin{document}

% =============================================================
% ======================== PROBLEM 10.2 =======================
% =============================================================

\subsection*{Problem 10.2}

Alice and Bob use the Diffie-Hellman key exchange technique with a common 
prime $q = 23$ and a primitive root $\alpha = 5$.

\begin{description}

% ==== PART A ====
\item[(a)] If Bob has a public key $Y_B = 10$, what is Bob's private key $X_B$?

% ==== SOLUTION ====
\begin{solution}
The equivalence $Y_B = \alpha^{X_B} \tmod{23}$ implies we need to solve 
$10 = 5^{X_B} \tmod{23}$ for Bob's private key. Some trial and error shows 
that $X_B = 3$.
\end{solution}

% ==== PART B ====
\item[(b)] If Alice has a public key $Y_A = 8$, what is the shared key $K$ 
with Bob?

% ==== SOLUTION ====
\begin{solution}
The shared key is computed as shown below.
	\[ K = (Y_A)^{X_B} = 8^3 \equiv 6 \tmod{23} \]
\end{solution}

% ==== PART C ====
\item[(c)] Show that 5 is a primitive root of 23.

% ==== SOLUTION ====
\begin{solution}
To show 5 is a primitive root of 23, we need to show the order of 5 is 22. 
This is verified in the summarized table below.
\begin{center}
\begin{tabular}{ c | *{7}{c} }
	$i$ & 1 & 2 & 3 & 4 & $\dots$ & 21 & 22 \\
	\hline
	$5^i \pmod{23}$ & 5 & 2 & 10 & 4 & $\dots$ & 14 & 1
\end{tabular}
\end{center}
\end{solution}

\end{description}


% =============================================================
% ======================== PROBLEM 10.6 =======================
% =============================================================

\subsection*{Problem 10.6}

Suppose Alice and Bob use an Elgamal scheme with a common prime $q = 157$ and 
a primitive root $\alpha = 5$.

\begin{description}

% ==== PART A ====
\item[(a)] If Bob has public key $Y_B = 10$ and Alice chose the random 
integer $k = 3$, what is the ciphertext of $M = 9$?

% ==== SOLUTION ====
\begin{solution}
First compute the one-time key $K = (Y_B)^k \tmod{q}$, or rather, 
$K = 10^3 \equiv 58 \tmod{157}$. Then, compute
	\[ C_1 = \alpha^k \equiv 125 \tmod{157} 
	\quad \text{ and } \quad
	C_2 = KM \equiv 51 \tmod{157} \]
and encrypt $M$ as the pair $(125, 51)$.
\end{solution}

% ==== PART B ====
\item[(b)] If Alice now chooses a different value of $k$ so that the encoding 
of $M = 9$ is $C = (25, C_2)$, what is the integer $C_2$?

% ==== SOLUTION ====
\begin{solution}
Know that $C_1 = 25 = 5^k \tmod{157}$ where it's clear that $k = 2$. This 
allows us to compute \newline
$K = 10^2 \equiv 100 \tmod{157}$ and thus 
$C_2 = 100 \cdot 9 \equiv 115 \tmod{157}$.
\end{solution}

\end{description}


\newpage

% =============================================================
% ======================= PROBLEM 10.11 =======================
% =============================================================

\subsection*{Problem 10.11}

Does the elliptic curve equation $y^2 = x^3 + x + 2$ define a group over 
$\mathbb{Z}_7$?

% ==== SOLUTION ====
\begin{solution}
Notice that in the elliptic curve, $a = 1$ and $b = 2$. If the equation 
defines a group, we need to verify $4a^3 + 27b^2 \neq 0$. Well,
	\[ 4(1)^3 + 27(2)^2 \equiv 0 \tmod{7} \]
and hence the elliptic curve does not define a group over $\mathbb{Z}_7$.
\end{solution}


% =============================================================
% ======================= PROBLEM 10.12 =======================
% =============================================================

\subsection*{Problem 10.12}

Consider the elliptic curve $E_7(2,1)$; that is, the curve is defined by 
$y^2 = x^3 + 2x + 1$ with a modulus of $p = 7$. Determine all of the points 
in $E_7(2, 1)$. Hint: Start by calculating the right-hand side of the 
equation for all values of $x$.

% ==== SOLUTION ====
\begin{solution}
Let $f(x) = x^3 + 2x + 1$ and $g(y) = y^2$. Compute 

\begin{minipage}{0.45\textwidth}
\centering
\begin{tabular}{ c | *{7}{c} }
				$x$ & 0 & 1 & 2 & 3 & 4 & 5 & 6 \\ \hline
	$f(x) \tmod{7}$ & 1 & 4 & 6 & 6 & 3 & 3 & 5
\end{tabular}
\end{minipage}
\begin{minipage}{0.5\textwidth}
\centering
\begin{tabular}{ c | *{7}{c} }
				$y$ & 0 & 1 & 2 & 3 & 4 & 5 & 6 \\ \hline
	$g(y) \tmod{7}$ & 0 & 1 & 4 & 2 & 2 & 4 & 1
\end{tabular}
\end{minipage}

If a point is to lie on $E_7(2,1)$, then $f(x) = g(y)$. The only outputs from 
both tables that match are 1 and 4. Thus, all the points on $E_7(2,1)$ are 
$(0,1), (0,6), (1,2),$ and $(1,5)$.
\end{solution}


\newpage


% =============================================================
% ======================= PROBLEM 10.15 =======================
% =============================================================

\subsection*{Problem 10.15}

This problem performs elliptic curve encryption/decryption using the scheme 
outlined in Section 10.4. The cryptosystem parameters are $E_{11}(1, 7)$ and 
$G = (3,2)$. B's private key is $n_B = 7$.

\begin{description}

% ==== PART A ====
\item[(a)] Find B's public key $P_B$.

% ==== SOLUTION ====
\begin{solution}
User $B$'s public key is determined by $7G$. Start by implementing addition 
over $E_{11}(1,7)$ as shown in the python script 
\purpText{scripts/ec\_add.py}.

% VSCode snippet.
\lstinputlisting[style = myPyStyle]{ec_add.py}

One way to compute $7G$ is to first compute $2G = G + G$, then 
$4G = 2G + 2G$, then $6G = 4G + 2G$ and finally $7G = 6G + G = (6,8)$.
\end{solution}

% ==== PART B ====
\item[(b)] A wishes to encrypt the message $P_m = (10,7)$ and chooses the 
random value $k = 5$. Determine the ciphertext $C_m$.

% ==== SOLUTION ====
\begin{solution}
To encrypt $P_m = (10,7)$, user A must compute the pair of points
	\[ C_m = \{kG, P_m + kP_B\} \]
Well 
	\[ kG = 5 (3,2) = (4,8)
	\qquad \text{and} \qquad
	P_m + kP_B = (10,7) + 5(6,8) = (1,8) \]

Thus, A encrypts $P_m$ as $C_m = \{(4,8),(1,8)\}$.
\end{solution}

% ==== PART C ====
\item[(c)] Show the calculation by which B recovers $P_m$ from $C_m$.

% ==== SOLUTION ====
\begin{solution}
User B recovers $P_m$ as follows
	\[ (P_m + kP_B) - n_B(kG) \; 
	= \; (1,8) - 7(4,8) \; 
	= \; (1,8) - (4,8) \; 
	= \; (10,7) \; = \; P_m \]
\end{solution}

\end{description}


\end{document}
