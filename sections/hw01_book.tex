\documentclass[../hw_sols.tex]{subfiles}
\setlist[description]{style = unboxed, leftmargin = 0.55cm}

\begin{document}

% =============================================================
% ======================== PROBLEM 3.1 ========================
% =============================================================

\subsection*{Problem 3.1}

A generalization of the Caesar cipher, known as the affine Caesar cipher, has 
the following form: For each plaintext letter $p$, substitute the ciphertext 
letter $C$:
	\[ C = E([a,b], p) = (ap + b) \tmod{26} \]
A basic requirement of any encryption algorithm is that it be one-to-one. 
That is, if $p \neq q$, then $E(k,p) \neq E(k,q)$. Otherwise, decryption is 
impossible, because more than one plaintext character maps into the same 
ciphertext character. The affine Caesar cipher is not one-to-one for all 
values of $a$. For example, for $a = 2$ and $b = 3$, then \newline
$E([a,b], 0) = E([a,b], 13) = 3$.

\begin{description}

% ==== PART A ====
\item[a.] Are there any limitations on the value of $b$? Explain why or why 
not.

% ==== SOLUTION ====
\begin{solution}
Addition by $b$ represents a shift substitution. Therefore, impose 
$b \in \mathbb{Z}_{26}$ so each shift is unique.
\end{solution}

% ==== PART B ====
\item[b.] Determine which values of $a$ are not allowed.

% ==== SOLUTION ====
\begin{solution}
The decryption algorithm is easy to construct.
	\[ D([a,b],C) := a^{-1}(C-b) \tmod{26} = p \]
For $a^{-1} \tmod{26}$ to exists, we require $\gcd(a,26) = 1$. Hence, $a$ 
cannot be even or 13.
\end{solution}

\end{description}


\newpage


% =============================================================
% ======================== PROBLEM 3.8 ========================
% =============================================================

\subsection*{Problem 3.8}

A disadvantage of the general monoalphabetic cipher is that both sender and 
receiver must commit the permuted cipher sequence to memory. A common 
technnique for avoiding this is to use a keyword from which the cipher 
sequence can be generated. For example, using the keyword CRYPTO, write out 
the keyword followed by unused letters in normal order and match this against 
the plaintext letters:
\begin{Verbatim}
plain:  a b c d e f g h i j k l m n o p q r s t u v w x y z
CIPHER: C R Y P T O A B D E F G H I J K L M N Q S U V W X Z
\end{Verbatim}

If it is felt that this process does not produce sufficient mixing, write the 
remaining letters on successive lines and then generate the sequence by 
reading down the columns:
\begin{Verbatim}[xleftmargin=3cm]
C R Y P T O
A B D E F G
H I J K L M
N Q S U V W
X Z
\end{Verbatim}

This yields the sequence:
\begin{Verbatim}
C A H N X R B I Q Z Y D J S P E K U T F L V O G M W
\end{Verbatim}

Such a system is used in the example in Section 3.2 of \cite{textbook} 
(the one that begins "it was disclosed yesterday"). Determine the keyword.

% ==== SOLUTION =====
\begin{solution}

From the decrypt, we know the correspondence below:
\begin{Verbatim}
plain:  a b c d e f g h i j k l m n o p q r s t u v w x y z
CIPHER: S A H V P B J W U ? ? X T D M Y ? E O Z I F Q ? G ?	
\end{Verbatim}

The keyword must have a length of 6 so that \verb|A| and \verb|B| 
line up horizontally.

\begin{minipage}{0.2\linewidth}
\begin{Verbatim}
S P U T ? I ?
A B ? D E F G
H J ? M O Q ?
V W X Y Z    
\end{Verbatim}
\end{minipage}
\qquad {\Huge $\Rightarrow$} \qquad
\begin{minipage}{0.2\linewidth}
\begin{Verbatim}
S P U T N I K
A B C D E F G
H J L M O Q R
V W X Y Z
\end{Verbatim}
\end{minipage}

With little effort, we see the keyword is \verb|SPUTNIK|.

\end{solution}


\newpage


% =============================================================
% ======================== PROBLEM 3.9 ========================
% =============================================================

\subsection*{Problem 3.9}

When the PT-109 American patrol boat, under the command of Lieutenant John F. 
Kennedy, was sunk by a Japanese destroyer, a message was received at an 
Australian wireless station in Playfair code:
\begin{Verbatim}[xleftmargin=3cm]
KXJEY UREBE ZWEHE WRYTU HEYFS
KREHE GOYFI WTTTU OLKSY CAJPO
BOTEI ZONTX BYBNT GONEY CUZWR
GDSON SXBOU YWRHE BAAHY USEDQ
\end{Verbatim}

The key used was \textit{royal new zealand navy}. Decrypt the 
message. Translate \verb|TT| into tt.

% ==== SOLUTION =====
\begin{solution}

Begin by creating the $5 \times 5$ Playfair matrix from the known keyword.
\begin{center}
\begin{tabular}{ | c | c | c | c | c | }
	\hline \rowcolor{cyan!20}
	R & O & Y & A & L \\
	\hline \rowcolor{cyan!20}
	N & E & W & Z & D \\
	\hline
	\cellcolor{cyan!20} V & B & C & F & G \\
	\hline
	H & I/J & K & M & P \\
	\hline
	Q &   S & T & U & X \\
	\hline
\end{tabular}
\end{center}

Then, we can break up the ciphertext into digrams and decrypt each 
by applying the algorithm in reverse.
% Playfair decrypt.
\VerbatimInput{verbatim/hw01/3_9_decrypt}

When rearranging the cipher, we can ignore extra x's to get the 
plaintext:
\begin{Verbatim}
pt boat one owe nine lost in action in blackett strait two miles
sw meresu cove crew of twelve request any information
\end{Verbatim}

\end{solution}


\newpage


% =============================================================
% ======================== PROBLEM 3.11 =======================
% =============================================================

\subsection*{Problem 3.11}

\begin{description}

% ==== PART A ====
\item[a.] Using this Playfair matrix:
\begin{center}
\begin{tabular}{ | c | c | c | c | c | }
	\hline
	M & F & H & I/J & K \\
	\hline
	U & N & O &   P & Q \\
	\hline
	Z & V & W &   X & Y \\
	\hline
	E & L & A &   R & G \\
	\hline
	D & S & T &   B & C \\
	\hline
\end{tabular}
\end{center}

Encrypt this message:
\begin{center}
	Must see you over Cadogan West. Coming at once.
\end{center}
\textit{Note:} This message is from the Sherlock Holmes story, The 
Adventure of the Bruce-Partingon plans.

% ==== SOLUTION =====
\begin{solution}
Split the plaintext into digrams and add pad with an 'x' to encrypt as shown 
below.
\begin{Verbatim}
plain:  mu st se ey ou ov er ca do ga nw es tc om in ga to nc ex
CIPHER: UZ TB DL GZ PN NW LG TG TU ER VO LD BD UH FP ER HW QS RZ
\end{Verbatim}
\end{solution}

% ==== PART B ====
\item[b.] Repeat part \textbf{a.} using the keyword \textit{largest}.

% ==== SOLUTION ====
\begin{solution}

Using the keyword \textit{largest}, we construct the Playfair matrix below 
and encrypt.
\begin{center}
\begin{tabular}{ | c | c | c | c | c | }
	\hline \rowcolor{cyan!20}
	L & A &   R & G & E \\
	\hline
	S \cellcolor{cyan!20} & T \cellcolor{cyan!20} & B & C & D \\
	\hline
	F & H & I/J & K & M \\
	\hline
	N & O &   P & Q & U \\
	\hline
	V & W &   X & Y & Z \\
	\hline
\end{tabular}
\end{center}

\begin{Verbatim}
CIPHER: UZ TB DL GZ PN NW LG TG TU ER OV DL BD UH PF ER HW QS RZ
\end{Verbatim}

\end{solution}

% ==== PART C ====
\item[c.] How do you account for the results of this problem? Can you 
generalize your conclusion?

% ==== SOLUTION ====
\begin{solution}
\newline  % Avoids inline error.
\begin{minipage}{0.55\linewidth}
The Playfair matrix can be thought of as a torus $\mathcal{T}$ by folding 
and gluing the vertical edges to each other and the same with the horizontal 
edges. The figure to the right shows the two Playfair matrices from parts 
\textbf{a} and \textbf{b}, outlined in red and blue. Notice that they both 
generate $\mathcal{T}$. It makes sense that our ciphertexts for parts 
\textbf{a} and \textbf{b} were the same. If we wanted different ciphertexts, 
we'd need to swap rows or columns of the Playfair matrix. This wouldn't be 
equivalent to any shift and thus providing a new ciphertext.
\end{minipage}
\quad
\begin{minipage}{0.4\linewidth}
\begin{tikzpicture}
	\matrix [matrix of math nodes] (m) {
		Y & Z & V & W &   X & Y & Z & V & W \\
		G & E & L & A &   R & G & E & L & A \\
		C & D & S & T &   B & C & D & S & T \\
		K & M & F & H & I/J & K & M & F & H \\
		Q & U & N & O &   P & Q & U & N & O \\
		Y & Z & V & W &   X & Y & Z & V & W \\
		G & E & L & A &   R & G & E & L & A \\
		C & D & S & T &   B & C & D & S & T \\
		K & M & F & H & I/J & K & M & F & H \\
	};

	% Draw outlines.
	\node[draw=red, thick, fit=(m-2-3)(m-6-7), inner sep=0pt] {};
	\node[draw=cyan, thick, fit=(m-4-2)(m-8-6), inner sep=0pt] {};
\end{tikzpicture}
\end{minipage}

\end{solution}

\end{description}


\newpage


% =============================================================
% ======================== PROBLEM 3.14 =======================
% =============================================================

\subsection*{Problem 3.14}

\begin{description}

% ==== PART A ====
\item[a.] Encrypt the message "meet me at the usual place at ten rather than 
eight o clock" using the Hill cipher with the key 
$\begin{bmatrix} 7 & 3 \\ 2 & 5 \end{bmatrix}$. 
Show your calculations and the result.

% ==== SOLUTION ====
\begin{solution}

Since the Hill cipher matrix $K \in \mathbb{Z}^{2 \times 2}$, we'll rewrite the 
plaintext into digrams and convert to numeric. \newline
We'll also pad the string with 'x' at the end to achieve an even length.
\begin{Verbatim}
plain: me et me at th eu su al pl ac ea tt en ra th er th an ei gh to cl oc kx
\end{Verbatim}

This gives us the following list of vectors $p_i$ for $1 \leq i \leq 24$:
\begin{center}
\begin{tabular}{ *{8}{ r } }
	(12,4) & (4,19) & (12,4) & (0,19) & (19,7) & (4,20) & (18,20) & (0,11) \\
	(15,11) & (0,2) & (4,0) & (19,19) & (4,13) & (17,0) & (19,7) & (4,17) \\
	(19,7) & (0,13) & (4,8) & (6,7) & (19,14) & (2,11) & (14,2) & (10,23)
\end{tabular}
\end{center}

To encrypt, let $p_i$ be the $i^{\text{th}}$ row of matrix $P$ so that
\begin{align*}
	PK =& 
	\begin{bmatrix}
		12 & 4 & 12 &  0 & 19 & \dots & 14 & 10 \\
		 4 & 19 & 4 & 19 &  7 & \dots &  2 & 23
	\end{bmatrix}^T
	\begin{bmatrix} 7 & 3 \\ 2 & 5 \end{bmatrix} \tmod{26} \\
	\equiv& 
	\begin{bmatrix}
		14 & 14 & 14 & 12 & 17 & \dots & 24 & 12 \\
		 4 &  3 &  4 & 17 & 14 & \dots &  0 & 15
	\end{bmatrix}^T 
	= \; C.
\end{align*}

We now convert each row $c_i$ of $C$ to alphabet characters to get the 
ciphertext below.
\begin{Verbatim}
CIPHER: OE OD OE MR QI KY WD XW EK CM PW CZ PZ RO AN SA EB FX KJ YA MP
\end{Verbatim}

\end{solution}

% ==== PART B ====
\item[b.] Show the calculations for the corresponding decryption of the 
ciphertext to recover the original plaintext.

% ==== SOLUTION ====
\begin{solution}

For decryption, compute $\det(K) = 29 \equiv 3 \tmod{26}$. Since 
$9 \equiv 3^{-1} \tmod{26}$, we have
	\[ K^{-1} 
	= 9 \begin{bmatrix} 5 & -3 \\ -2 & 7 \end{bmatrix} 
	\equiv \begin{bmatrix} 19 & 25 \\ 8 & 11 \end{bmatrix} \tmod{26} \]

To obtain the plaintext, we simply take the product	
\begin{align*}
	CK^{-1} =& 
	\begin{bmatrix}
		14 & 14 & 14 & 12 & 17 & \dots & 24 & 12 \\
		4 & 3 & 4 & 17 & 14 & \dots & 0 & 15
	\end{bmatrix}^T
	\begin{bmatrix} 19 & 25 \\ 8 & 11 \end{bmatrix} \tmod{26} \\
	\equiv& 
	\begin{bmatrix}
		12 & 4 & 12 & 0 & 19 & \dots & 14 & 10 \\
		4 & 19 & 4 & 19 & 7 & \dots & 2 & 23
	\end{bmatrix}
	= P
\end{align*}

This recovers plaintext matrix $P$ from part \textbf{a}.

\end{solution}

\end{description}


\newpage


% =============================================================
% ======================== PROBLEM 3.18 =======================
% =============================================================

\subsection*{Problem 3.18}

\begin{description}

% ==== PART B ====
\item[b.] Determine the inverse mod 26 of
	\[ \begin{bmatrix}
		6  & 24 &  1 \\ 
		13 & 16 & 10 \\ 
		20 & 17 & 15
	\end{bmatrix} \]

% ==== SOLUTION ====
\begin{solution}

Denote matrix $A$. To compute $A^{-1} \tmod{26}$, we first need the 
determinant.
	\[ \det(A) = 441 \equiv 25 \tmod{26} \]

We can then find the cofactors of 
$A^T = 
\begin{bmatrix}
	 6 & 13 & 20 \\ 
	24 & 16 & 17 \\ 
	 1 & 10 & 15
\end{bmatrix}$
as shown below.

\begin{multicols}{3}
	$C_{1,1} = (-1)^2(16 \cdot 15 - 10 \cdot 17) \\
	 C_{2,1} = (-1)^3(13 \cdot 15 - 10 \cdot 20)  \\
	 C_{3,1} = (-1)^4(13 \cdot 17 - \cdot 16 \cdot 20)$
	
	$C_{1,2} = (-1)^3(24 \cdot 15 - 1 \cdot 17) \\
	 C_{2,2} = (-1)^4(6 \cdot 15 - 1 \cdot 20)   \\
	 C_{3,2} = (-1)^5(6 \cdot 17 - 24 \cdot 20)$
	
	$C_{1,3} = (-1)^4(24 \cdot 10 - 1 \cdot 16) \\
	 C_{2,3} = (-1)^5(6 \cdot 10 - 1 \cdot 13)   \\
	 C_{3,3} = (-1)^6(6 \cdot 16 - 24 \cdot 13)$
\end{multicols}

This allows us to create the adjoint matrix
	\[ \text{adj}(A) =
	\begin{bmatrix}
		 70 & -343 &  224 \\ 
		  5 &   70 &  -47 \\ 
		-99 &  378 & -216
	\end{bmatrix} \]

Lastly, notice that $25 \cdot 25 \equiv 1 \tmod{26}$ so we take the product
	\[ (\det(A))^{-1} \cdot \text{adj}(A) 
	= 25 \cdot \text{adj}(A) 
	\equiv 
	\begin{bmatrix}
		 8 &  5 & 10 \\ 
		21 &  8 & 21 \\ 
		21 & 12 & 8
	\end{bmatrix} \tmod{26} 
	= A^{-1} \]

\end{solution}

\end{description}


\newpage


% =============================================================
% ======================== PROBLEM 3.19 =======================
% =============================================================

\subsection*{Problem 3.19}

Using the Vigenère cipher, encrypt the word "explanation" using the word 
"leg".

% ==== SOLUTION ====
\begin{solution}

Begin by repeating the keyword over the plaintext as shown below.
\begin{Verbatim}
key:    legleglegle
plain:  explanation
\end{Verbatim}

We can then perform the required shifts by taking the sum mod 26.
\begin{center}
\begin{tabular}{ *{12}{| c } | }
	\hline
		key & 11 &  4 &  6 & 11 & 4 &  6 & 11 &  4 &  6 & 11 &  4 \\
	\hline
		plain &  4 & 23 & 15 & 11 & 0 & 13 &  0 & 19 &  8 & 14 & 13 \\
	\hline
	CIPHER & 15 &  1 & 21 & 22 & 4 & 19 & 11 & 23 & 14 & 25 & 17 \\
	\hline
\end{tabular}
\end{center}

Encryption is completed by writing the numerical values into an alphabetic 
ciphertext.
\begin{Verbatim}
CIPHER: PBVWETLXOZR
\end{Verbatim}

\end{solution}


\end{document}